%% scribe appears to not distinguish ru and rū —— confirm
\documentclass[12pt]{book}

% Set paper size and margins using geometry
\usepackage[paperwidth=170mm, paperheight=240mm, inner=30mm, outer=25mm, top=25mm, bottom=30mm]{geometry}

\usepackage{microtype}
\usepackage[pdfusetitle,hidelinks]{hyperref}
\usepackage{setspace}
\usepackage{color,soul}
\emergencystretch=0.5em % allow more dynamic spacing between words when there is an overflow, i.e. prevents hyphenation problems

\usepackage[series={A,B},noend,noeledsec,noledgroup]{reledmac}
% \arrangementX[B]{twocol}
% \hsizetwocolX[B]{.48\hsize}
% \colalignX[B]{\justified}
\renewcommand*{\thefootnoteA}{\roman{footnoteA}}

\usepackage{fancyhdr}
\pagestyle{fancy}}
\fancyhf{} % Clear all header and footer fields
\fancyfoot[C]{\thepage} % Place page number in center of footer
\renewcommand{\headrulewidth}{0pt} % Remove the header line

\usepackage{marginnote}
\usepackage{longtable}

\usepackage{fontspec}
\usepackage{polyglossia} 		
	\setmainlanguage[script=latin]{sanskrit}
	\setotherlanguage{english}	
	\setmainfont[Ligatures=TeX]{Libertinus Serif}
	\newfontfamily\sanskritfont[Ligatures=TeX]{Libertinus Serif}
	\newfontfamily\mantrafont[Ligatures=TeX]{STIX Two Text}[Scale=MatchLowercase]
	\newfontfamily\devfont[Script=Devanagari]{Noto Sans Devanagari}


% make brackets etc. never in italics
% must be turned off before printing bibliography
\usepackage{embrac}
\providecommand\textsi[1]{#1} % To avoid error in xelatex looking for textsi command
\AddOpEmph{|}
\AddOpEmph{/}

%%% Bibliography
\usepackage[authordate16,backend=biber]{biblatex-chicago}
\addbibresource{~/Documents/personal/bibliography.bibtex}
\renewcommand{\mkbibnamefamily}[1]{\textsc{#1}}
\renewcommand\postnotedelim{\addcolon\addspace} % adds colon after year in citation.

\DeclareBibliographyCategory{fullcited}
\newcommand{\mybibexclude}[1]{\addtocategory{fullcited}{#1}}


\newcommand{\crux} {\hspace{0em}\textsuperscript{†}\hspace{0em}}
\newcommand{\emdash} {\hspace{0em}—\hspace{0em}}

\title{Tattvaratnāvaloka\\ and its Vivaraṇa}
\author{Vāgīśvarakīrti}

\setcounter{secnumdepth}{2}
\renewcommand{\thesection}{}
\renewcommand{\thesubsection}{\arabic{subsection}}

% Increase the depth of sectioning in the table of contents
\setcounter{secnumdepth}{4} % Numbering depth
\setcounter{tocdepth}{4}    % Table of contents depth

% Define \subsubsubsection
\makeatletter
\newcounter{subsubsubsection}[subsubsection] % Create a counter for subsubsubsections
\renewcommand\thesubsubsubsection{\thesubsubsection.\arabic{subsubsubsection}} % Define numbering

\newcommand\subsubsubsection{\@startsection{subsubsubsection}{4}{\z@}%
  {-3.25ex \@plus -1ex \@minus -.2ex}%
  {1.5ex \@plus .2ex}%
  {\normalfont\normalsize\bfseries}}
\newcommand\subsubsubsectionmark[1]{}
\makeatother

% Adjust formatting (optional)
\usepackage{titlesec}
\titleformat{\subsubsubsection}[runin]
  {\normalfont\normalsize\bfseries}{\thesubsubsubsection}{1em}{}
\begin{document}
\maketitle

% Define a new intercharacter class for Sanskrit question marks
\makeatletter
\newXeTeXintercharclass\noextraclass
\XeTeXcharclass `\? = \noextraclass
\XeTeXcharclass `\! = \noextraclass
\XeTeXcharclass `\; = \noextraclass
\XeTeXcharclass `\: = \noextraclass

% Do not italicise / and |
\AddOpEmph{|}
\AddOpEmph{/}

% Remove the extra space between Sanskrit text and ? or !
\XeTeXinterchartoks 0 \noextraclass = {\nobreak}
\XeTeXinterchartoks \noextraclass 0 = {\nobreak}
\makeatother

%sigla
\newcommand{\PCreading}{$^{pc}$}
\newcommand{\ACreading}{$^{ac}$}
\newcommand{\MS}{K}
\newcommand{\EDD}{E\textsubscript{DH}}
\newcommand{\TM}{TM\textsubscript{D}}
\newcommand{\TVA}{TVA\textsubscript{D}}
\newcommand{\TVB}{TVB\textsubscript{G}}
\newcommand{\TIB}{TV}
\newcommand{\sigmareading}[1]{$\Sigma$\textsubscript{#1}}

% App shortcuts
\newcommand{\emd} {\emph{em.}}
\newcommand{\conj} {\emph{conj.}}
\newcommand{\possibleemd} {\emph{possible em.}}
\newcommand{\possibleconj} {\emph{possible conj.}}
\newcommand{\corr} {\emph{corr.}}
\newcommand{\diag} {\emph{diag.\ conj.}}

\section*{Some Conventions/Policies}
\begin{itemize}
    \item Tibetan translations are included in the apparatus when they indicate variant Sanskrit readings.
    
    \item Sanskrit renderings suggested by Tibetan appear in brackets after the translation's siglum. These renderings are hypothetical and cannot be provided in all cases.
    
    \item When two Tibetan translations differ slightly, they are separated by a semicolon. Only the second translation includes a Sanskrit rendering.
    
    \item A Tibetan translation's siglum is included when it appears to support one of multiple Sanskrit readings, based on the editors' judgment.
    
    \item Tibetan is not included in the apparatus when it offers no clear support for or against a reading.

	\item When both Tibetan translations of the commentary agree, they are given the siglum \TIB .
\end{itemize}

\section*{Some Things to Check}
\begin{itemize}
	\item Consistency in \TIB\ in rendering \emph{vijñāna} and \emph{jñāna}. 

	\item Consistency in \TIB\ in rendering \emph{iti}s. 
\end{itemize}
\newpage
\section*{Sigla and Abbreviations}
\noindent\begin{longtable}{ l p{0.7\linewidth} }
	\noindent TaRaa & Tattvaratnāvaloka\\

	\noindent TaRaa-Vi & Tattvaratnāvalokavivaraṇa\\

	\noindent \EDD\ & Dhīḥ vol. 21, pp.\ 129–149.\\

	\noindent \MS\ & NAK 5–252 = NGMPP A 915/4\\

	\noindent \TM & \emph{De kho na nyid rin po che snang ba}. Tōhoku no.\ 1889. sDe dge bstan 'gyur, vol.\ Pi, fols.\ 203r3–204r5. Tr.\ by 'Gos Lhas btsas\\

	\noindent \TVA & \emph{De kho na nyid rin po che snang ba'i rnam par bshad pa}.  Tōh.\ 1890. sDe dge bsTan 'gyur, vol.\ 44 (rGyud 'grel, Pi), fols.\ 204r5–214v4. Tr.\ by 'Gos Lhas btsas.\\

	\noindent \TVB & \emph{De kho na nyid rin po che snang ba'i rnam par bshad pa}. Ōtani no.\ 4793. bsTan 'gyur gSer bris ma, vol.\ 84 (83 in BDRC outline(?)), (rGyud 'brel, Zhu), fols. 70v–85v. translator given.\\

	\noindent \TIB & Both Tibetan translations of the commentary (differences, if any, indicated in a mini-aparatus)
\end{longtable}

\noindent\begin{longtable}{ l p{0.7\linewidth} }
$ac$ & \emph{ante correctionem} \\
\emph{deest} & omitted in \\
\diag & diagnostic conjecture [e.g.\ `reconstructed' from Tibetan]\\
\conj & conjecture\\
\emd & emendation [an emendation is made with a high degree of confidence, whereas a conjecture proposes a correction while acknowledging a greater possibility for alternatives]\\
fol./fols. & folio/folios \\
$pc$ & \emph{post correctionem} \\
$r$ & recto \\
$v$ & verso \\
$\Sigma$\textsubscript{X} & Reading shared in all witnesses but X \\
((kiṃcit)) & Reading uncertain—either illegible or otherwise in doubt \\
<kiṃcit> & Reading cancelled \\
\crux kiṃcit\crux & Reading does not make sense to the editor and an adequate conjecture was not able to be chosen. \\
{[}kiṃcit{]} & Indication of a diagnostic conjecture  \\
.. & Damaged \emph{akṣara} (one . per half \emph{akṣara}) \\
... & Lacunae of an unknown quantity of \emph{akṣara}s \\
° & Mark of abbreviation \\
\end{longtable}

\section*{Text}
\subsection{maṅgalācaraṇam}
\begin{quote}
	[\MS\ fol.\ 1r] [siddhaṃ]\footnoteB{
		[siddhaṃ]] \MS ; oṁ \EDD
	} namaḥ śrīsadgurupādebhyaḥ |\footnoteA{
		Scribal homage
	}
	
%	[\TM\ fol.\ 203r]  || rgya gar skad du | tattv'a ratna a'a lo ka | bod skad du | de kho na nyid rin po che snang ba | bcom ldan 'das 'jam pa'i rdo rje la phyag 'tshal lo || 

	anupamasukharūpī śrīnivāso 'nivāso \\
	nirupamadaśadevīrūpavidyaḥ\footnoteB{
		nirupama°] \EDD ; nirūpama° \MS
	} savidyaḥ |\\
	tribhuvanahitasaukhyaprāptikāro 'vikāro \\
	jayati kamalapāṇir yāvad āśāvikāśāḥ || 1 ||\footnoteA{
		This verse is in Mālinī metre. % LLLLLLGGGLGGLGG X 4
	}
\end{quote}

\medskip\noindent [\MS\ fol.\ 2r3] namaḥ samantakāyavākcittavajrāya.\footnoteA{
	Scribal homage
}\\

\noindent anupametyādi.
kamalaṃ padmaṃ pāṇau yasya sa kamalapāṇir avalokiteśvaro bhagavāñ jayatīti sambandhaḥ.
kiṃviśiṣṭaḥ?
anupamam ity\footnoteB{
	kiṃviśiṣṭaḥ? anupamam ity] \MS\ \EDD ; khyad par ji lta bu zhig dang ldan zhe na | dpe med ces bya ba la sogs pa smos te | dpe med pa ni \TVA ; khyad par ji lta bu zhig dang ldan zhe na | dpe med ces bya ba la sogs pa smos te | dpe med pa dang \TVB\ (kiṃviśiṣṭa ity āha anupamam ityādi. anupamam)
}\footnoteA{
	Here one may wish to conjecture a reading such as, \emph{anumapetyādi. anupamam ity} \ldots 
	This reading is partially suggested by \TIB : \emph{khyad par ji lta bu zhig dang ldan zhe na | dpe med ces bya ba la sogs pa smos te | dpe med pa ni} (\emph{ni}] \TVA ; \emph{dang} \TVB).
	The corruption, if there is one, can be seen as a kind of haplography.
	The text nonetheless reads acceptably well with the transmitted reading (although perhaps less smoothly); thus we feel that while a conjecture is possible it is not strongly compelling.
} atipraṇītatvamahattvāsaṃsārasthāyitvalakṣaṇair\footnoteB{
	°saṃsārasthāyitva°] \MS ; °saṃsārasthāyisva° \EDD
} dharmair yuktasyānyasyābhāvād upamārahitaṃ sukham eva rūpaṃ svabhāvo yasya sa tathoktaḥ.
punar api kiṃviśiṣṭaḥ?
śrīḥ puṇyajñānasambhāralakṣaṇā, tasyā nivāsa āśrayo yaḥ sa tathā.
dharmakāyarūpatvena\footnoteB{
	dharmakāyarūpatvena] \emd ; dharmakāyarūpitvena \MS\ \EDD
}\footnoteA{
	The manuscript's \emph{dharmakāyarūpitvena} is theoretically acceptable and nearly synonymous; however, \emph{dharmakāyarūpatvena} is more expected, with forms in \emph{°rūpatvena} being vastly more frequent in Classical Sanskrit. \TIB's reading \emph{chos kyi sku'i ngo bo nyid kyis} does not clearly confirm either variant, as \emph{rūpin} in the root verse is also translated as \emph{ngo bo}. We provisionally adopt \emph{dharmakāyarūpatvena}, but cannot fully discount the transmitted reading.
} sarvagatatvāt [\EDD\ p.\ 132] pratiniyatanivāsābhāvād anivāsaḥ.

punaḥ kīdṛśaḥ?
nirupamāḥ paramarūpayauvanaśṛṅgārādirasamahākaruṇādiyuktatvenopamātikrāntā rūpavajrāditārāparyantadaśadevīrūpā vidyāḥ paricārakatvena\footnoteB{
	paricārakatvena] \emd ; sapari((c))ārakatvena \MS ; saparivārakatvena \EDD
} yasya sa tathā.
saha svābhārūpayā vidyayā\footnoteB{
	vidyayā] \MS\ \EDD ; rig pa ste | shes rab \TIB\ (vidyayā prajñayā)
} vartata iti savidyaḥ.
tribhuvanasya tribhuvanavartino janasya yad dhitam āyatipathyaṃ\footnoteB{
	āyatipathyaṃ] \emph{variant word division in} \EDD : āyati pathyaṃ; \emph{and in} \MS : āyati | pathyaṃ
}\footnoteA{
	We need not necessarily read a compound for \emph{āyatipathyaṃ}, treating instead \emph{āyati} as a locative of \emph{āyat}.
	The expression appears as a gloss for \emph{hita} in several Buddhists texts, such as in Vilāsavajra's \emph{Nāmamantrārthāvalokinī}: \emph{mahyaṃ hitaṃ maddhitaṃ hitam āyatipathyam āgāmipariṇāmatvāt} (p.\ 233).
	Similarly, Durvekamiśra writes in his \emph{Hetubinduṭīkāloka}: \emph{parasmai hitam āyati pathyaṃ} (p.\ 212). 
	In both cases, the construction is ambiguous, but in the latter case, the editors of Durvekamiśra's text have not taken it as a compound.

	\hspace*{1em}The word \emph{āyatipathya} is used less ambiguously in compound by Śākyarakṣita, quoted in the following note. Similarly, the roughly parallel expression \emph{āyatisukha} is evidnetly treated as a compound by Yaśomitra in his \emph{Abhidharmakośavyākhyā}: \emph{aihikasukhārtham apuṇyam iti. ihasukhāpekṣayā tat kṛtaṃ nāyatisukhāpekṣayety arthaḥ} (vol.\ 1 p.\ 299).
	Note also the contrast made with \emph{aihikasukha}.
	Likewise, we can find a compounded form of \emph{āyatiduḥkha} in a verse attributed to Naradatta in the \emph{Subhāṣitaratnakośa}: \emph{muṇḍāpriyād āyatiduḥkhadāyino vasantam utsārya vijṛmbhitaśriyaḥ | na kaḥ khalāt tāpitamitramaṇḍalād upaiti pāpaṃ tapavāsarād iva ||} `Who does not become miserable because of a rogue who, like a hot day, is hated by widows (? \emph{muṇḍā}) (the hot day being hated by bald men), who leads to future pain, whose wealth expands after he expels those living with him (like the hot day manifests its richness having dismissed the spring), and who annoys his circle of friends (like on a hot day the orb of the sun is heated) (cf.\ \cite[553]{ingalls1965}).
} buddhatvādikaṃ, saukhyaṃ tadātve pathyaṃ\footnoteB{
	saukhyaṃ tadātve pathyaṃ] \conj ; tad dāpayati pathyaṃ \MS\ \EDD\ (\emph{word division unclear}); de la bde ba ni 'phral gyi phan pa \TVA ; de la bde ba ni bde ba ste \TVB
} cakravartitvādikam,\footnoteA{
	The text is insecure here but perhaps not far from the author's intention.
	Where the manuscript reads \emph{tad dāpayati pathyaṃ} (word division unclear, \emph{pa} and \emph{ya} touching), we conjecture \emph{tadātve pathyaṃ}, following only partially the lead of \TVA . 
	The Tibetan translations read as follows: \emph{gang la phan pa ni ma 'ongs pa'i phan pa ste | sangs rgyas nyid la sogs pa'o || de la bde ba ni 'phral gyi phan pa ste |} (\TVA); \emph{gang la phan pa ni ma 'ongs pa'i phan pa ste | sangs rgyas nyid la sogs pa dang | de la bde ba ni bde ba ste | 'khor lo bsgyur ba nyid la sogs pa'o ||} (\TVB).
	It appears that \TVB\ also transmits a corrupt reading with \emph{de la bde ba nit bde ba ste}.
	\TVA\ suggests reading something that contrasts with \emph{āyatipathyaṃ}, for which \emph{tadātve pathyaṃ} fits.
	Another possibility is \emph{āpātapathyaṃ}, but \emph{tadātva} is more often used in contrast with \emph{āyati}.
	See, for example, Śākyarakṣita's \emph{Vṛttamālāstutivṛtti}: \emph{pṛthagjanatve 'pi āyatipathyadarśinas tadātve ca niṣpāpāḥ} (p.\ 299); `Although ordinary people, they see the future welfare and are without sin in the present moment.'
	% Compare also Cakrapāṇidatta's \emph{Āyurvedadīpikā}: \emph{hitam eveti āyativiśuddham eva tadātve duḥkhakaram api | priyam eveti tadātve sukhakaram āyativiruddham ||}.

	\hspace*{1em} The Tibetan translations also suggest that \emph{hita} and \emph{saukhya} are linked with relative and corelative pronouns: \emph{gang la} and \emph{de la}, or \emph{yasya} and \emph{tasya} in Sanskrit.
	This does not yield good sense.
	It is possible that \emph{tadātve} was misread by the translator as a corelative pronoun, while it is also possible that a second relative pronoun (\emph{yat}) or a conjunction (\emph{ca}) was found in the original text near \emph{saukhyaṃ}.
	Here \emph{hitasaukhya} within the larger compound is only really viable as a \emph{dvandva}: Avalokiteśvara causes the attainment of (ultimate) welfare and (temporary) happiness for all beings.
	Given that, \emph{tayor yā prāptiḥ} might be preferable to \emph{tasya yā prāptiḥ}, but the singular is also probably acceptable in place of the dual.
} tasya yā prāptiḥ\footnoteB{
	prāptiḥ] \MS\ \EDD ; thob pa ni rnyed pa ste \TIB\ (prāptir lābhaḥ)
} [\MS\ fol.\ 2v] sākṣātkriyā, tasyāḥ karaṇaṃ kāro yasya sa tathā.\footnoteA{
	It is notable that Vāgīśvarakīrti evidently understands \emph{°prāptikāra} as a \emph{bahuvrīhi}, whereas other commentators may prefer to treat it akin to \emph{kumbhakāra} and therefore as an \emph{upapadasamāsa} as per \emph{Aṣṭādhyāyī} 2.2.19 (\emph{upapadam atiṅ}).
	Given the latter understanding, the expected gloss for \emph{prāptikāra} would be \emph{prāptiṃ karoti}.
	Compounds ending in \emph{kāra} are occasionally analysed as \emph{ṣaṣṭhītatpuruṣa}s: see, for examples, Vijñāneśvara's \emph{Mitākṣarā} ad \emph{Yājñavalkya-dharmaśāstra} 2.61 on \emph{satyaṃkārakṛta}, here referring roughly to a thing `acquired with a pledge', i.e., acquired as earnest money: \emph{karaṇaṃ kāraḥ, bhāve ghañ. satyasya kāraḥ satyaṃkāraḥ—kāre satyāgadasya (Aṣṭādhyāyī 6.3.70) iti mum. satyaṃkāreṇa kṛtaṃ satyaṃkārakṛtam} (p.\ 275).
	We are unable to provide another example of a compound ending in \emph{kāra} analysed as a \emph{bahuvrīhi}, but we should also note that the compound \emph{prāptikāra} is itself rare.

	\hspace*{1em}\TIB\ does not clearly reflect a \emph{ṣaṣṭhībahuvrīhi} analysis, nor does it very clearly point to another reading: \emph{de dag sgrub par mdzad po gang yin pa de la de skad ces bya'o} (\TVA); \emph{de dag gi rgyu mdzad pa gang yin pa de la de skad ces bya'o} (\TVB).
}
aparinirvāṇadharmakatvenāpratiṣṭhitanirvāṇarūpatvenā\footnoteB{
	°rūpatvenā°] \MS\ \EDD ; ngo bo rnyed pas \TVA ; ngo bo brnyed pas \TVB\ (°rūpaprāptyā°)
}nyathātva\-la\-kṣa\-ṇa\-sya vikārasyābhāvād avikāraḥ.
evaṃviśiṣṭo bhagavāñ jayati.
 
kiyantaṃ kālam ity āha\emdash yāvad āśāvikāśāḥ. āśā daśa diśo gaganasvarūpāḥ. yadvā āśāḥ sarvasattvānāṃ bhavabhogatṛṣṇāḥ.\footnoteB{
	°tṛṣṇāḥ] \EDD\ (°tṛṣṇās); tṛṣṇā \MS
} tāsāṃ vikāśā avakāśāḥ pravartanāni, prādurbhāvā iti yāvat.
te yāvat tāvad\footnoteB{
	te yāvat tāvad] \emd ; tā yāvat tāvad \MS\ \EDD ; de srid du \TIB\ (tāvad)
} bhagavāñ jayati, sarvahariharahiraṇyagarbhādibhyaḥ prakṛṣṭo bhavatīty arthaḥ.

atrānupamasukharūpīty anena svārthasaṃpattiḥ kathitā.
śrīnivāsa ity anena tadupāyaḥ, puṇyajñānasambhārayoḥ śrīśadbenābhihitatvāt.
tribhuvanahitasaukhyaprāptikāra ity anena parārthasaṃpattir uktā.
nirupamadaśadevīrūpavidyaḥ savidya ity anena tadupāyaḥ, \footnoteB{
	tathābhūta°] \MS\ \EDD\ \TVB\ (\emph{de lta bu}); \emph{no reflex in} \TVA
}\hspace{0em}tathābhūtadaśadevīdvātriṃśallakṣaṇāśītyanuvyañjanakāyākāraśūnyena\footnoteB{
	°kāyā°] \MS\ \EDD ; dam pa'i sku \TIB\ (°satkāyā°)
} sarvākāraparārthasaṃpatteḥ kartum aśakyatvād iti.

\subsection{prayojanādi}
\begin{quote}
	śrīmantranītigatacārucaturthaseka-\\
	rūpaṃ vidanti na hi ye sphuṭaśabdaśūnyam |\\
	nānopadeśagaṇasaṃkulasaptabhedais\\
	teṣāṃ sphuṭāvagataye kriyate prayatnaḥ || 2 ||\footnoteA{
		This verse is in Vasantatilakā.
	}
\end{quote}

\noindent śrīmantranītiśabdena\footnoteB{
	śrīmantranītiśabdena] \MS\ \EDD\ \TVB\ (dpal ldan sngags kyi gzhung lugs zhes bya ba'i sgras); dpal ldan sngags kyi gzhung lugs shes || zhes bya ba la sogs pa la | sngags kyi gzhung lugs zhes bya ba'i sgras ni | \TVA\ (śrīmantranītigatetyādi. mantranītiśabdena)
} sāmānyayogatantravācakenāpi śrīsamājaḥ\footnoteB{
	śrīsamājaḥ] \MS\ \EDD ; shugs kyis dpal gsang ba 'dus pa \TIB\ (sāmarthyāt śrīsamājaḥ)
} parigṛhyate, caturthārthakasyānyatrāsambhavāt.
śeṣaṃ subodham.
nānācāryopadeśagaṇasaṃkulai\hspace{0em}[\EDD\ p.\ 133]\hspace{0em}r vyākulaiḥ\footnoteB{
	vyākulaiḥ] \MS\ \EDD ; rnam par dkrugs pas rnam pa thams cad la rnam par khyab pa \TVA ; rnam par 'khrugs pa rnam par bkab pa ste \TVB ; vyākulair vipūrṇaiḥ \possibleconj\ (\emph{see notes})%
} saptabhir bhedaiḥ prakārair\footnoteB{
	prakārair] \MS\ \EDD ; \emph{no reflex in} \TIB
} atītānāgatavartamānācārya\footnoteB{
	°vartamānā°] \EDD ; °pravartamānā° \MS
}gatopadeśarāśi\-saṃ\-grā\-hakaiḥ.\footnoteB{
	°gato°] \MS\ \EDD\ \TVB\ (gtogs pa); \emph{no reflex in} \TVA
}\footnoteA{
	In this case \TVB\ resembles closely the Sanskrit text transmitted in \MS , apart from the addition of a further gloss after \emph{vyākula}.
	The reading \emph{rnam par bkab pa} (`covered') doesn't yield much sense, but it could be a mistake for \emph{rnam par bkang ba} (`filled'), which is perfectly fitting and synonymous with \TVA 's \emph{rnam par khyab pa} (Negi records the latter as rendering \emph{vipūrṇa} in some texts).
	One may wish to conjecture such a reading.
	\TVA\ is significantly different here, even though most of the words of the transmitted Sanskrit text are still reflected: \emph{du ma'i man ngag ces bya ba la sogs pa la | 'das pa dang ma 'ongs pa dang | da ltar gyi slob dpon du ma'i man ngag gi tshogs yang dag par bsdus pa'i mdun gyi dbye bas yongs su dkrugs pa ni | rnam par dkrugs pas rnam pa thams cad la rnam par khyab pa ste | des bsgrub par bya ba dkrugs pa'o}{ ||}
	The text is dubious but reflects a Sanskrit text along the following lines: \emph{nānopadeśetyādi. atītānāgatavartamānācāryopadeśarāśisaṃgrāhakaiḥ saptabhir bhedaiḥ saṃkulair vyākulaiḥ sarvatravīpūrṇaiḥ taiḥ sādhyasaṃkulaiḥ}.
}
sphuṭāvagataye sukhena sphuṭapratītyartham\footnoteB{
	sukhena sphuṭapratītyartham] \MS\ \EDD ; bde bar gnas par khong du chud par bya ba'i phyir \TVA ; bde bar gsal bar khong du chud par bya'o \TVB
} iti.

\subsection{tīrthikānāṃ tattvaṃ sādhyaṃ ca}
\begin{quote}
	sambhrāntabodhā nikhilā hi tīrthyās \\% final s in tīrthyās is unclear, but probably there; there is a slight crease in the image
	tattvasya sādhyasya ca rūpavittau |\\
	tebhyaḥ prakṛṣṭaḥ kila tattvavettā\\
	vedāntavādīti janapravādaḥ || 3 ||\footnoteA{
		This verse is in Indravajrā.
	}
\end{quote}

\noindent sambhrāntetyādi.
sambhrānto vibhrānto bodhaḥ prajñāviśeṣo yeṣāṃ tīrthikānāṃ te tatho[\MS\ fol.\ 3r]ktāḥ.\footnoteB{
	te tathoktāḥ] \MS\PCreading ; te thoktāḥ \MS\ACreading ; tathoktāḥ \EDD
}
sarva eva tīrthyā ātmātmīyagrahatimiropahatabuddhinayanāḥ.
tattvam idam iti sādhyam idam\footnoteB{
	sādhyam idam] \emd ; sādhyaṃ cedam \MS\ \EDD
} iti ca tattvasya sādhyasya yat\footnoteB{
	yat] \EDD\ (\emd); tat \MS
} svarūpam, tasya yā vittiḥ pratītiḥ, tasyāṃ bhrāntāḥ.
śeṣaṃ subodham.

% § 3_2
nanu tattvasādhyayor upādeyatvenaikarūpatvāt kathaṃ tattvasya sādhyasya ceti\footnoteB{
	kathaṃ tattvasya sādhyasya ceti] \emd ; tat kathaṃ tatvasya sādhyasya ceti \MS ; tattvasya sādhyasya ceti kathaṃ \EDD\ (\emd)
}\footnoteA{
	\EDD\ misreads the manuscript as \emph{tattvasya sādhyasya ceti} and supplies \emph{kathaṃ} after \emph{ceti}.
	There is in fact a \emph{kathaṃ} before \emph{tattvasya} in the manuscript, but the \emph{tat} preceding that \emph{kathaṃ} is evidently a corruption.
} bhedena nirdeśa iti cet.
asad etat.
tattvaṃ hy upādeyatve 'pi\footnoteB{
	upādeyatve 'pi] \conj\ (\TIB : blang bar bya ba nyid yin yang); upādeyatvenāpi \MS\ \EDD
} sukhaduḥkhopekṣādisakalapratibhāsasaṃdohavyāpakam.\footnoteB{
	°vyāpakam] \MS\ (°kaṃ) \EDD\ \TVB\ (khyab par byed pa yin la); shes bya tsam du khyab par byed pa yin la \TVA\ (°vyāpakaṃ jñeyamātratvena)
}
sādhyaṃ cānabhimataparihāreṇecchālakṣaṇaṃ phalam upādeyatve 'pi sakalaprāṇibhir avaśyam evāsādhyavyāvṛttyā sādhayitavyatvenābhimatam ity adoṣaḥ.

\subsection{vedāntavādināṃ śrāvakapratyekabuddhānāṃ ca sādhyāni}
% § 4_1
tatra tāvad\footnoteB{
	tāvad] \MS\ \EDD\ \TVA\ (re zhig); \emph{no reflex in} \TVB
} vedāntavādyabhimataṃ sādhyam āha\emdash ānandarūpam ityādi.

\begin{quote}
	ānandarūpaṃ svavid\footnoteA{
		From the commentary it is clear that \emph{svavid} is not in compound; thus, being an accusative form of a feminine noun, we expect \emph{svavidam}.
		The form may be grammatically justifiable if it is treated as neuter adjective, akin to \emph{vedavid}.
	} aprakampyaṃ \\
	vedāntinaḥ sādhyam uṣanti śāntam\footnoteB{
		śāntam] \corr ; sāntam \MS\ \EDD ; \emph{no reflex in} \TM
	} |\\
	saśrāvakāḥ\footnoteB{
		saśrāvakāḥ] \emd ; saśrāvakā \MS\ \EDD
	} khaḍgajināś ca sādhyam\\
	icchanti rūpādyupadher virāmam || 4 ||
\end{quote}

% § 4_2
\noindent ānandarūpam iti sadā sukhamayatvāt.
svavid iti jyotīrūpatvena\footnoteB{
	jyotīrūpatvena] \MS ; jyotirūpatvena \EDD
} svayaṃ prakāśamānatvāt.\footnoteB{
	prakāśamānatvāt] \EDD\ (\emd); prakāśamānāt \MS
}
aprakampyam iti nityatayā\footnoteB{
	nityatayā] \EDD ; anityatayā \MS\ \TIB\ (mi rtag pa nyid kyis)
} kampayitum aśakyatvāt.
śāntam\footnoteB{
	śāntam] \corr ; sāntam \MS\ \EDD
} iti kleśopakleśaśūnyatvena parikalpitatvāt.
evaṃvidhaṃ sādhyam uṣanti kāmayante.

% § 4_3
saha śrāvakair vartante ye khaḍgajināḥ khaḍgaviṣāṇakalpā ekacāriṇo vargacāriṇaś\footnoteB{
	vargacāriṇaś] \MS\ \TIB\ (tshogs kyi spyod pa); vanacāriṇaś \EDD 
} ca pratyekabuddhās te sādhyam icchanti.
kīdṛśam?
rūpādyupadher virāmaṃ rūpavedanāsaṃjñāsaṃskāravijñānalakṣaṇānām upadhīnāṃ skandhānāṃ virāmaṃ vicchedam, nirodham iti yāvat.
[\EDD\ p.\ 134] etad uktaṃ bhavati\emdash sarvaśrāvakapratyekabuddhāḥ sopadhiśeṣanirupadhiśeṣabhedena bhinne 'pi nirvāṇe\footnoteB{
	nirvāṇe] \EDD ; nirvāṇa° \MS
} nirupadhiśeṣam eva nirvāṇaṃ sā[\MS\ fol.\ 3v]kṣātkarta\-vya\-tve-na sādhyaṃ pratipannāḥ.

\subsection{pāramitānayavādināṃ caturvidhaṃ sādhyam}
idānīṃ pāramitānayavādinām abhimataṃ\footnoteB{
	abhimataṃ] \EDD; abhimata \MS
} caturvidhaṃ sādhyam āha\emdash ākāraśūnyam ityādi.

\begin{quote}
	ākāraśūnyaṃ gaganendurūpaṃ \\
	pratyātmavedyaṃ karuṇārasaṃ ca |\\
	sallakṣaṇair bhūṣitam\footnoteB{
		bhūṣitam] \EDD ; bhuṣitam \MS
	} arthakāri \\
	dānādiniṣyandam apetasaukhyam || 5 ||
 
	sānandasallakṣaṇamaṇḍitāṅgaṃ \\
	sambhujyamānaṃ daśabhūmisaṃsthaiḥ |\\
	sattvārthakāri pravadanti sādhyaṃ \\
	dānādiṣaṭpāramitānayasthāḥ || 6 ||\footnoteA{
		These two verses are in Indravajrā.
	}
\end{quote}

\subsubsection{pāramitānaye prathamaṃ sādhyam}
\noindent ākārair nīlapītasukhaduḥkhādibhiś citrarūpaiḥ śūnyaṃ nirākāram.
ata eva gaganasyeva nirākāratvenendor iva prabhāsvaratvena rūpaṃ svabhāvo yasya tat tathā.
pratyātmavedyam iti svasaṃvedanaikavedyam.\footnoteB{
	svasaṃvedanaikavedyam] \EDD\ (\emd) (°vedyaṃ); svasaṃvedyanaikavedyaṃ \MS
}
karuṇā duḥkhād\footnoteB{
	karuṇā duḥkhād] \MS ; karuṇāduḥkhā° \EDD
} duḥkhahetor vā sakalajagadabhyuddharaṇakāmatā.\footnoteB{
	°abhyuddharaṇakāmatā] \emd ; °atyuddharaṇakāmatā \MS\ \EDD
}\footnoteA{
	An alternative to \emph{°abhyuddharaṇakāmatā} is to read \emph{°samuddharaṇakāmatā}.
	This definition of \emph{karuṇā}, in various forms, is well known in Buddhist texts.
	See, for instance, Durvekamiśra's \emph{Hetubinduṭīkāloka}: \emph{... duḥkhāt duḥkhahetor vā samuddharaṇakāmatā nāma yā karuṇā ...} (p. 234); or Manorathānandin's \emph{Pramāṇavārttikavṛtti}: \emph{duḥkhād duḥkhahetoś ca samuddharaṇakāmatā karuṇā} (edition reads \emph{dukhā°}; p. 21).
}
saiva rasaḥ svabhāvo yasya tat tathoktam.
etad uktaṃ bhavati\emdash nīlapītādicitrākāraśūnyaṃ nirābhāsaṃ\footnoteB{
	nirābhāsaṃ] \emd ; nirābhāsa° \MS\ \EDD
} nirañjanaṃ\footnoteA{
	One may instead wish to accept the manuscript reading \emph{nirābhāsanirañjanaṃ}, which is understandable as a \emph{viśeṣaṇasamāsa}.
	The combination of \emph{nirābhāsaṃ nirañjanam} occurrs in a verse from an untraced source cited in Raviśrījñāna's \emph{Amṛtakaṇikā}: \emph{yat kāyaṃ sarvabuddhānāṃ nirābhāsaṃ nirañjanam | ajñātam akṛtaṃ śuddham abhāvādivivarjitam ||} (p. 19)
	% see paramagambhīra sādhana in Vajrayoginī sādhana codex
} gaganopamaṃ svacchaṃ sakalajagadarthakāri\footnoteA{
	\emph{sakalajagadarthakāri} can also be read in compound with \emph{mahākaruṇā°}. This is reflected in \TIB : \emph{'gro ba ma lus pa'i don byed pa'i snying rje chen po}.
	Regardless, the two are evidenly closely related.
} mahākaruṇāyuktaṃ pratyātmavedyaṃ pāramitopadeśaśabdābhidheyaṃ sādhyam iti pāramitānaye prathamaṃ sādhyam.

% § 5.2
\subsubsection{pāramitānaye dvitīyaṃ sādhyam}
\noindent śobhanāni ca tāni lakṣaṇāni ca dvātriṃśallakṣaṇasaṃjñakāni,\footnoteB{
	dvātriṃśallakṣaṇasaṃjñakāni] \conj ; dvātriṃśallakṣaṇasaṃjñakāni ceti \MS\ \EDD ; mdzes pa'i mtshan sum cu rtsa gnyis zhes bya ste \TIB\ (dvātriṃśatsallakṣaṇānīti / dvātriṃśatsallakṣaṇasaṃjñakāni)
}\footnoteA{
	The manuscript reading \emph{ceti} after \emph{dvātriṃśallakṣaṇasaṃjñakāni} appear superfluous.
	The commentary analyses \emph{sallakṣaṇa} as a \emph{karmadhāraya}, glossing \emph{sat} with \emph{śobhana}; \emph{dvātriṃśallakṣaṇa} serves as a clarification of that, requiring no further conjunction.
	Likewise, the words \emph{iti} and \emph{saṃjñaka} together are redundant. 
	In \TIB , the \emph{zhes bya} following the phrase may either render \emph{iti} or \emph{saṃjñaka}—we find this rendering for the latter in the commentary on verse 9 for \emph{mahāsukhasaṃjñaka}. 
	We cannot fully discount that Vāgīśvarakīrti wrote the transmitted reading, nor can we give a clear explanation for the corruption, if it is one.
	Nonetheless, given that this appears to be genuine redundancy rather than simply a stylistic oddity, we provisionally conjecture a slightly smoother reading.
} tair bhūṣitam.
arthaṃ janānāṃ prayojanaṃ kartuṃ śīlaṃ svabhāvo yasya tad arthakāri.\footnoteB{
	tad arthakāri] \MS\ \EDD\ \TVA\ (de ni don mdzad pa'o); de ni de'i don mdzad pa'o \TVB\ (tad tadarthakāri)
}
dānādīnāṃ daśapāramitānāṃ niṣyandaṃ\footnoteA{
	Here \emph{niṣyandaṃ} should be understood either as an accusative form (as it is in the verse) or (less likely) anomalously as a neuter noun.
} tatprakarṣaprabhavatvena sadṛśaṃ phalam.\footnoteA{
	cf.\ \emph{Abhidharmakośa} 2.57c: \emph{niṣyando hetusadṛśaḥ}. 
	Vāgīśvarakīrti perhaps also alludes to Dharmakīrti's definition of yogic perception in \emph{Nyāyabindu} 11: \emph{bhūtārthabhāvanāprakarṣaparyantajaṃ yogijñānaṃ ceti}.
}
duḥkhasya pūrvam eva prahīṇatvāt, sākṣātkaraṇāvasthāyāṃ\footnoteB{
	sākṣātkaraṇāvasthāyāṃ] \conj\ (\textsc{Isaacson}); sākṣātkṛtāvasthāyāṃ \EDD ; sākṣātkṛtāvatāsthāyāṃ \MS
}\footnoteA{
	\textsc{Isaacson} (personal communication) proposes \emph{sākṣātkaraṇāvasthāyāṃ} or \emph{sākṣātkṛtyāvasthāyāṃ} as potentially supperior readings to the manuscript's \emph{sākṣātkṛtāvatāsthāyāṃ} or the previous edition's \emph{sākṣātkṛtāvasthāyāṃ}.

	\hspace*{1em} In support of the former, see Vāgīśvarakīrti's \emph{Saṃkṣiptābhiṣekavidhi}: \emph{tadanantaram ekatathatāmatena tayaiva bhinnamate tv ānayā svasaṃviditajñānasākṣātkaraṇāvasthāyāṃ pūrvoktagāthayā adhyeṣitavate śiṣyāya tatpāṇau tasyāḥ pāṇiṃ pratisthāpya |} (p. 417)
} saukhyasyāpy abhāvāt,\footnoteB{
	abhāvāt] \emd\ (\textsc{Isaacson}); abhāvatvāt \MS\ \EDD
} upekṣārūpatvenāpetasaukhyam apagatasaukhyam.
etad uktaṃ bhavati\emdash dvātriṃśallakṣaṇadharāśītyanuvyañjanavirājitaśarīraṃ sakalajagadarthakāri dānādipāramitābhyāsa\crux balenātmānaṃ\footnoteB{
	°balenātmānaṃ] \MS\ \EDD ; stobs kyis bdag nyid \TVA ; stobs kyis byung ba \TVB
}\crux\ samyaksaṃbuddharūpaṃ sukhaduḥkharahitatvenopekṣārūpaṃ dvitīyaṃ sādhyam.


\subsubsection{pāramitānaye tṛtīyaṃ sādhyam}
[\EDD\ p.\ 135] sānandetyādi.
sahānandena vartata iti sā[\MS\ fol.\ 4r]\-na\-ndam.
sānandaṃ ca tat sallakṣaṇamaṇḍitāṅgaṃ ca\footnoteB{
	sallakṣaṇamaṇḍitāṅgaṃ ca] \emd\ (\textsc{Isaacson}); sallakṣaṇamaṇḍitāṅgaṃ \MS\ \EDD
} sambhujyamānaṃ dharmadeśanādvāreṇopajīvyamānam.\footnoteB{
	°opajīvyamānam] \MS\ \EDD ; nye bar longs spyod par gyur pa'o \TIB\ (°opabhujyamānam)
}\footnoteA{
	For \emph{upajīvyamāna} we might expect \emph{nye bar 'tsho ba} in Tibetan.
	Below \emph{upabhujyamāna} is translated as \emph{longs spyod par bya ba} and then \emph{nye bar longs spyod par bya ba}.
}
kaiḥ?
daśabhūmīśvaraiḥ, pariśiṣṭabhūmisthitānām\footnoteB{
	pariṣiṣṭabhūmi°] \corr ; pariṣiṣṭa bhumi° \EDD
} agocaratvāt.
daśabhūmiprāptair avalokiteśvaramañjuśrīprabhṛtibhir upabhujyamānam iti yāvat.
etad uktaṃ bhavati\emdash śuddhāvāsopari ghanavyūhasaṃjñake\footnoteB{
	°saṃjñake] \emd ; °saṃjñako \MS ; °saṃjñakaḥ \EDD\ (\emd)
} samyaksaṃbuddhabhuvane yathā bhagavān ānandarūpaḥ sambhogakāyātmā nirmāṇadvāreṇa\footnoteB{
	nirmāṇadvāreṇa] \MS\ \EDD ; sprul pa'i sku'i sgo nas \TIB\ (nirmāṇakāyadvāreṇa)
} sakalajagadarthasaṃpādakaḥ śrāvakapratyekabuddhanavabhūmīśvarair apy adṛśyaśarīro daśabhūmīśvarair eva paraṃ bodhisattvair\footnoteB{
	paraṃ bodhisattvair] \MS\ (°satvair) \EDD ; mchog tu gyur pa'i byang chub sems dpa' \TIB\ (paramabodhisattvair)
} dharmaśravaṇadvāreṇopabhujyamāna\footnoteB{
	°bhujyamāna] \emd ; °bhujyamānam \MS\ \EDD
} āsaṃsāraṃ cakāsti, tathaiva tat sādhyam iti tṛtīyam.

\subsubsection{pāramitānaye caturthaṃ sādhyam}
\begin{quote}
	saṃpūrya dānādiguṇān aśeṣān \\
	saṃbuddhakṛtyaṃ\footnoteB{
		saṃbuddhakṛtyaṃ] \emd\ (\emph{cf.} TaRaA-V: saṃbuddhānāṃ \ldots\ avaśyakartavyaṃ kṛtsnaṃ); saṃbuddhya kṛtyaṃ \MS\ \EDD
	} sakalaṃ ca kṛtvā |\\
	yad bhūtakoṭeḥ karaṇaṃ ca sākṣāt \\
	sādhyaṃ tad apy asti nirodharūpam || 7 ||\footnoteA{
		This verse is in Indravajrā metre.
	}	
\end{quote}

\noindent saṃpūryetyādi.
dānādipāramitā eva guṇā, guṇyante\footnoteA{
	In the \emph{Dhātupāṭha}, the tenth class verbal root \emph{√guṇa} is said to express \emph{āmantraṇa}. Here, however, this is a denominative verb with the sense of \emph{āmreḍaṇa} (multiplication/repetition) formed from the noun \emph{guṇa}. 
} 'bhyasyanta iti kṛtvā.
tān saṃpūrya paripūrṇān\footnoteB{
	paripūrṇān] \emd ; paripūrṇaṃ \MS\ \EDD
} kṛtvā, yat saṃbuddhānāṃ kṛtyaṃ sakalam\footnoteB{
	kṛtyaṃ sakalam] \conj ; sakalam \MS\ \EDD ; \emph{no reflex in \TIB}
}\footnoteA{
	The manuscript's reading of simply \emph{sakalaṃ} instead of \emph{kṛtyaṃ sakalam} is asymmetrical given the following gloss, \emph{avaśyakartavyaṃ kṛtsnaṃ}. Here \TIB\ reads simply \emph{nges par mdzad par bya ba ma lus pa}, reflecting only the gloss and neither \emph{sakalam} of \MS\ nor the conjecture \emph{kṛtyaṃ sakalam}. It is also possible that \emph{sakalam} is a mistaken scribal addition, but it's also possible that even if the Tibetan translators saw \emph{kṛtyaṃ sakalam}, they chose not to render this because of the superfluous sounding result in Tibetan.
	We believe the manuscript's transmitted reading is improbable.
} avaśyakartavyaṃ kṛtsnaṃ tad api kṛtvā, bhūtakoṭeḥ śūnyatālakṣaṇāyāś cittacaittanirodhātmikāyā\footnoteB{
	cittacaitta°] \EDD\ (\emd); cittacaitya° \MS
} yat sākṣāt karaṇaṃ tad api sādhyam astīti pāramitānayasthā evaṃ bruvate caturthaṃ sādhyam iti.

\subsection{mantranaye saptavidhaṃ sādhyam}
% §6_1
\subsubsection{mantranaye prathamaṃ sādhyam}
idānīṃ mantranayopadiṣṭaṃ saptavidhaṃ\footnoteB{
	saptavidhaṃ] \EDD\ (\TM : rnam pa bdun); caturthaṃ \MS
} sādhyaṃ kathayitum āha\emdash svābhāṅganetyādi.

\begin{quote}
	svābhāṅganāśleṣi\footnoteB{
		svābhāṅganāśleṣi] \EDD\ (\corr); svābhāṅgaṇāśleṣi \MS
	} janārthakāri\footnoteB{
		janārthakāri] \conj\ (\TM : 'gro ba yi don mdzad; TaRaA-V: jagadarthakāri); ta..rthakāri \MS\ (\emph{akṣara uncertain, perhaps} gna \emph{or} mva); tadarthakāri \EDD
	} \\
	duḥkhaiḥ sukhaiś caiva vimuktirūpam |\\
	aśītyanuvyañjanabhūṣitāṅgam \\
	apetakalpaṃ pravadanti sādhyam || 8 ||\footnoteA{
		This verse is in Upajāti.
	}
	% GGLGGLLGLGL X 2
	% LGLGGLLGLGL X 2
\end{quote}

\noindent svābhāṅganām\footnoteB{
	svābhāṅganām] \EDD\ (\corr); svābhāṅgaṇām \MS
} āśleṣituṃ śīlaṃ svabhāvo yasya tat svābhāṅganāśleṣi.\footnoteB{
	svābhāṅganāśleṣi] \corr ; svābhāṅgaṇāśleṣi \MS\ \EDD
}
[\EDD\ p.\ 136] apetakalpaṃ vyapagatakalpam, kalpanārahitam iti yāvat.
anyat subodham.
ayam arthaḥ\emdash samāliṅgitasvābhāṅganāśleṣi jagadarthakāri\footnoteB{
	°svābhāṅganāśleṣi jagadarthakāri] \conj\ (\TVB : nyid dang mtshungs pa'i lha mos 'khyud pa can 'gro ba'i don mdzad pa); °svābhāṅganāśleṣajagadarthakāri \MS\ \EDD ; nyid dang mtshungs pa'i lha mos 'khyud pa can | 'gro ba ma lus pa'i don mdzad pa \TVA\ (°svābhāṅganāśleṣy aśeṣajagadarthakāri)
}\footnoteA{
	The compound \emph{°svābhāṅganāśleṣajagadarthakāri} is strictly speaking not impossible, and could perhaps be interpreted as an instrumental \emph{tatpuruṣa}; however, given that this is a prose explanation of the verse, there is no need for the author to use such a compound and it seems more likely that the scribe left off the \emph{ikāra}.
} dvātriṃśallakṣaṇavibhūṣitaśarīram\footnoteB{
	śarīram] \EDD ; śarīra \MS
} upekṣārūpaṃ\footnoteB{
	upekṣārūpaṃ] \MS\ \EDD ; btang snyoms kyi ngo bo du 'khor ba ji srid du bzhugs pa mngon du bya ba yin no zhe bya ba \TVA ; btang snyoms kyi ngo bo nyid du 'khor ba ji bzhugs pa mngon sum du bya ba yin zhes bya ba \TVB\ (upekṣārūpaṃ āsaṃsārasthāyi sākṣāt kriyata iti)
}\footnoteA{
	Something along the lines of \emph{āsaṃsārasthāyi sākṣāt kriyata iti} may have dropped out of the text here given \TIB , but there is no very compelling reason to think that it did.
	The additional words are relevant, given that it is a pertinent feature of the first \emph{sādhya} that it remains active for as long as \emph{saṃsāra} continues to exist.
	We can be reasonably sure that \TIB\ reflects \emph{āsaṃsārasthāyi} with \emph{'khor ba ji srid du bzhugs pa}, as this is the Tibetan rendering of this word in the next section.%
} prathamaṃ sādhyam.

\subsubsection{mantranaye dvitīyaṃ sādhyam}
\begin{quote}
	svadevatākāraviśeṣaśūnyaṃ \\
	prāg eva sambhāvya sukhaṃ sphuṭaṃ sat |\\
	mahāsukhākhyaṃ jagadarthakāri \\
	cintāmaṇiprakhyam uvāca kaścit || 9 ||\footnoteA{
		This verse is Viparītākhyānikī metre.
	}	
\end{quote}

\noindent svadevatetyādi.
svadevatākāraviśeṣeṇa\footnoteB{
	svadevatā°] \MS\ \EDD\ \TVB\ (rang lha'i); lha \TVA\ (devatā°)
} sveṣṭadevatākāreṇa \hspace{0.2em}śū\-nyam, nirākāram iti yāvat.
prāg eva prathamataram\footnoteB{
	prathamataram] \MS ; prathamataro° \EDD
} upadeśānantaram eva\footnoteB{
	upadeśānantaram eva] \EDD\ (\emd); upadeśāntaram eva \MS ; bshad ma thag pa'i \TIB
}\footnoteA{
	Normally \emph{bshad ma thag pa} in Tibetan has the sense of \emph{anantarokta}, but here the translator probably did intend it to render \emph{upadeśānantaram eva} as we find the same rendering later in the paragraph.
} devatākāranirapekṣaṃ sukhaṃ sambhāvya, bhāvanayā sākṣāt kṛtvā, sphuṭaṃ\footnoteB{
	sphuṭaṃ] \MS ; \emph{deest in} \EDD ; ma gsal ba \TIB
}\footnoteA{
	The understanding offered by \TIB , which reflects \emph{asphuṭaṃ} instead of \emph{sphuṭaṃ}, appears to indicate a misunderstanding on the translator's part, confusing the word division of \emph{kṛtvā sphuṭaṃ}.
	It is not possible for \emph{sphuṭīkṛtaṃ} to take an accusitve object, nor is a form such as \emph{sphuṭīkṛtya} possible without larger changes to the text.
} sphu[\MS\ fol.\ 4v]\hspace{0em}ṭīkṛtaṃ san mahāsukhasaṃjñakaṃ bhavati.
tac ca jagadarthakāri cintāmaṇisamānarūpam.
etad uktaṃ bhavati\emdash upadeśānantaram eva mantramudrādevatākārarahitaṃ\footnoteB{
	°rahitaṃ] \MS\ \EDD\ \TVB\ (spangs ste); spangs te | bde ba 'ba' zhig tsam \TVA\ (°rahitaṃ sukhamātra°)
} bhāvanayā sphuṭīkṛtaṃ mahāsukhasaṃjñakaṃ cintāmaṇivaj jagadarthakāri\footnoteB{
	jagadarthakāri] \MS\ \EDD\ \TVB\ ('gro ba'i don mdzad pa); 'gro ba ma lus pa'i don mdzad pa \TVA\ (aśeṣajagadarthakāri)
} māyopamam āsaṃsārasthāyi dvitīyaṃ sādhyam.

\subsubsection{mantranaye tṛtīyaṃ sādhyam}
\begin{quote}
	kṛtvā sākṣāt svādhipaṃ [\MS\ fol.\ 1v] sātarūpaṃ \\
	paścāt tyaktvā sātamātraṃ phalaṃ syāt |\\
	śuddhaṃ sākṣāc chakyate naiva kartuṃ \\
	tenākāro bhāvitaḥ svādhipasya || 10 ||\footnoteA{
		This verse is in Śālinī metre.
	}	
\end{quote}

\noindent kṛtvetyādi.
svādhipaṃ sveṣṭadaivataṃ sākṣāt kṛtvāmukhīkṛtya sātarūpaṃ sukhaikasvabhāvam, paścād devatākāraṃ parityajya, sukhamātraṃ\footnoteB{
	sukhamātraṃ] \emd ; sukhamātra° \MS\ \EDD
} phalaṃ sādhyaṃ vyavasthitaṃ syāt.

nanu yadi\footnoteB{
	nanu yadi] \conj ; nanu \MS\ \EDD ; gal te \TVA\ ([nanu] yadi); \emph{no clear reflex} \TVB
} sākṣāt kṛtvāpi devatākāras tyaktavyaḥ, tarhi prathamam eva kasmād [\EDD\ p.\ 137] vibhāvitaḥ?
sukhamātram eva dvitīyasādhyavat kiṃ na vibhāvitam?\footnoteB{
	vibhāvitam] \emd ; vibhāvitaḥ \EDD\ (\emd); vibhāgato \MS
}
kiṃ vṛthāprayāsenety\footnoteB{
	vṛthāprayāsenety] \EDD ; vyathāprayāsenety \MS
} āha\emdash śuddham ityādi.
śuddhaṃ kevalaṃ devatākāravirahitaṃ sukhamātraṃ naiva sākṣāt kartuṃ śakyate, ākārarahitasya sukhasyānupalambhāt.\footnoteB{
	sukhasyā°] \MS\ \EDD\ \TVB\ (bde ba); bde ba 'ba' zhig \TVA\ (kevalasukhasyā°) 
}
tasmāt tena kāraṇenākāro bhāvitaḥ svādhipasyeti tṛtīyam.\footnoteB{
	°eti tṛtīyam] \emd\ \TVB\ (zhes bya ba gsum pa yin no); °eti tṛtīyaḥ \MS\ \EDD ; ste bsgrub par bya ba gsum pa yin no \TVA\ (tṛtīyaṃ sādhyam)
}
ayam arthaḥ\footnoteB{
	arthaḥ] \EDD ; artha \MS
}\emdash devatākārasaṃvalitam eva sukhaṃ vibhāvya, sākṣādbhūte devatākāraṃ tyaktvā, sukhamātram eva sādhyam uktaguṇam.\footnoteA{
	Here \TIB\ reads \emph{yon tan du 'chad do}, whereas \MS\ transmits the reading \emph{uktaguṇam}.
	It is difficult to say if the Tibetan rendering represents a different underlying Sanskrit reading, but it does convey a different sense. Whereas the Tibetan seems to say that the \emph{sādhya} 'is taught to be a good quality', the Sanskrit suggests the meaning 'which has the previously mentioned qualities'.
}

% § 6.4
\subsubsection{mantranaye caturthaṃ sādhyam}
\begin{quote}
	gagaṇasamaśarīraṃ lakṣaṇair bhūṣitāṅgaṃ \\
	nirupamasukhapūrṇaṃ\footnoteB{
		nirupama°] \EDD ; nirūpama° \MS
	} svābhayā saṃgataṃ ca |\\
	sphuradamitamunīndraiḥ\footnoteB{
		°munīndraiḥ] \emd ; °munīndraḥ \MS\ \EDD
	} sarvasattvārthakāri \\
	pravadati punar anyaḥ sādhyam ucchedaśūnyam \mbox{|| 11 ||}\footnoteA{
		This verse is in Mālinī metre.
	}
\end{quote}

% §	TV6.4_1
\noindent gagaṇetyādi.
gagaṇasamaṃ māyopamaṃ vicārāsahaṃ\footnoteB{
	māyopamaṃ vicārāsahaṃ] \MS\ (\emph{slightly unclrear}); māyopamavicārasaha \EDD
} śarīraṃ yasya.
lakṣaṇair dvātriṃśadbhir\footnoteB{
	lakṣaṇair dvātriṃśadbhir] \MS\ \EDD ; mtshan gyi ste | mtshan sum cu rtsa gnyis \TVA ; mtshan gyis te | mtshan sum cu rtsa gnyis \TVB\ (lakṣaṇair [iti] dvātriṃśadbhir lakṣaṇair)
} aśītibhiś cānuvyañjanair maṇḍitāny aṅgāni yasya.
nirupamaiḥ sthaulya\footnoteB{
	sthaulya°] \MS\ \EDD ; rgya nom pa nyid dang | rgya che ba nyid dang \TVA\ (praṇītatvasthaulya°); lhun che ba nyid dang | \TVB\ (sthaulya°)
}-\hspace{0em}nairantarya\footnoteB{
	°nairantarya°] \EDD\ (\emd) (\TIB : bar med pa nyid dang); °nairuttaryā° \MS
}-\hspace{0em}āsaṃsārapravāhitva\footnoteB{
	°āsaṃsārapravāhitva°] \emd ; °āsaṃsāraṃpravāhitva° \EDD\ \MS
}-\hspace{0em}nirāsravatvādibhir upamābhāvād upamātikrāntaiḥ sukhaiḥ pūrṇaṃ romāgraparyantaṃ\footnoteB{
	pūrṇaṃ romāgraparyantaṃ] \conj\ (\TIB : gang ba ni | ba spu rtse mo'i mthar thug pa); pūrṇṇaṃ masimāgrapayantaṃ \MS ; pūrṇatāṃ samāśrayantaṃ \EDD
}\footnoteA{
	This conjecture follows the Tibetan translation, with the reading in \MS\ being difficult to account for.
	See, for instance, a similar expression in \emph{Siddhaikavīrasādhana} (author unknown): \emph{tato niḥsṛtaraśmibhir ā pādatalād vālāgraparyāntaprāptaṃ bhāvyate} (\emph{Sādhanamālā} no.\ 67, vol.\ 1, p.\ 67); \emph{de las byung ba'i 'od zer gyis rkang pa'i mthil nas skra'i rtse mo'i mthar thug pa khyab par bsgoms te} (Tōh.\ 3461 fol.\ 116r). ADD REFERENCE
} saṃpūrṇam.\footnoteA{
	TO CHECK: \TVA\ appears to be defective here, with different readings in Derge Koyosan and Delhi. \TVA : \emph{ba spu'i rtse mo'i mthar thug par gyur pa'o ||} (\emph{gyur pa'o ||}] Koyosan; \emph{gyur ba'i} Delhi [MW23703]). \TVB : \emph{ba spu'i rtse mo'i mthar thug par yang dag par gang bar gyur pa'o ||}
}
svābhayā ca tathābhūtayā saṃgataṃ samāliṅgitam.
sphuradbhir\footnoteB{
	sphuradbhir] \MS\ \EDD ; 'phro bar gyur pa de yang \TVA\ (sphuradbhir tair api) (\emph{other syntactic placement possible}); 'phro ba yang \TVB\ (sphurdbhair api)
} anantanirmitair munīndrais tathābhūtair eva sarvasattvārthakāri.\footnoteB{
	sarvasattvārtha°] \MS\ \EDD\ (\TVB : sems can thams cad kyi don); sems can gyi don \TVA\ (sattvārtha°)
}
ucchedeneti nirodhena śūnyaṃ tucchaṃ riktam.\footnoteB{
	tucchaṃ riktaṃ] \MS ; bhūsthaṃ riktam \EDD ; spangs pa’o \TIB\ (tucchaṃ / riktaṃ)
}
% For sthaulya°, see commentary on verse one. These are to be taken as qualities (dharmas) qualifying the sukha and thus making it unique.

% §	TV6.4_2
etad uktaṃ bhavati\emdash gagana-māyā-marīci\footnoteB{
	māyāmarīci°] \MS\ \EDD\ (\TVB : sgyu ma dang | smig rgyu dang |); sgyu ma dang | smig rgyu dang | smig rgyu dang | \TVA\ (māyāmarīcīndrajāla° / māyendrajālamarīci°)
}-\hspace{0em}gandharvanagara\hspace{0em}-\-udakacandra\hspace{0em}-\hspace{0em}pratibimba\--\hspace{0em}svapnopamam\footnoteB{
	°svapnopamam] \EDD ; svapnāpayaṃ \MS
}\footnoteB{
	\TVA\ adds an element to the list, perhaps \emph{indrajāla} in Sanskrit.
	The reading has the advantage of form a list of eight, but this precise list is otherwise unattested as a list of eight illusions.
} [\MS\ fol.\ 5r] ekānekabhāvābhāvagrāhyagrāhakasvabhāvarahitam anādyantam aśeṣavastusaṃdohasvabhāvam\footnoteB{
	anādyantam aśeṣavastusaṃdohasvabhāvam] \MS\ \EDD ; thog ma dang tha ma med pa'i dngos po ma lus pa'i rang bzhin \TIB\ (anādyantāśeṣavastusvabhāvam)
}\footnoteA{
	\TIB\ is perhaps ambiguous and may not reflect a different reading of the Sanskrit if \emph{thog ma dang tha ma med pa'i} is understood to qualify \emph{rang bzhin} instead of \emph{dngos po}.
} anābhāsaṃ nirañjanaṃ sarvopamātikrāntaṃ paramasūkṣmātigambhīraprajñārūpatayā dharmakāyasvabhāvam, dvātriṃśallakṣaṇavibhūṣitaśarīram aśītyanuvyañjanavirājitagātraṃ\footnoteB{
	°gātraṃ] \MS\ \EDD ; \emph{no reflext in} \TIB
} paramaśṛṅgārayauvanādyupetaṃ svābhāṅganāliṅgitāṅgaṃ rūpavajrāditārāparyantadevīgaṇair anantaprabhedānimittarati\footnoteB{
	anantaprabhedānimittarati°] \conj\ (\TVA : mtshan ma med pa'i dga' ba'i rang gi ngo bo'i rab tu dbye ba dpag tu med pas); anantaprabhedānimittārati° \MS \EDD ; mtshan ma med pa'i rang gi ngo bo'i rab tu dpag tu med pas \TVB
}\hspace{0em}svarūpaparamānandopabhogadvāreṇa\footnoteA{
	The compound beginning \emph{anantaprabheda°} is challenging to unpack and not entirely secure in its reading.
	\MS\ transmits the compound in a way that includes either the word \emph{arati} or \emph{ārati}, neither of which can reject \emph{prima facia}.
	\TVA\ suggests reading \emph{rati}, while \TVB\ has no reflex of the word but may be corrupt, given that it sounds rather incomplete. \TVB\ is also missing a reflex of \emph{bheda}, although it does have one of \emph{pra} from \emph{prabheda}, also indicating corruption.
	The term \emph{animittarati} or \emph{mtshan ma med pa'i dga' ba} does occurr in Jñānapāda's \emph{Samantabhadrasādhana}: \emph{animittarativiśuddheḥ samastadevīgaṇasvabhāvaṃ tat |} (122ab; reconstructed in \cite[261]{szantosaccone2023}); \emph{mtshan ma med pa'i dga' ba rnam dag pa | ma lus lha mo'i tshogs kyi ngo bo nyid ||} (Tōh.\ 1855 fol.\ 34r5; the translation in Tōh.\ 1856 by Smṛtijñānakīrti poses some problems and need not be dealt with here); `Because of purification by signless pleasure, that [awareness] has as its nature the group of all goddesses.'
	This parralel does lend support to reading \emph{animittarati}, but the context is technical and esoteric, so some caution is due.
	% Tōh. 1856, Smṛti's translation, appears to have mtshan ma med pa'i dag pa, probably an error. Perhaps should check various witnesses and report.

	As for the analysis of the compound, while various possibilities may be entertained, the main ambiguity is whether \emph{anantaprabheda} qualifies \emph{animittarati} or \emph{paramānanda}. An analysis on the basis of the former could read: \emph{anantāḥ prabhedāḥ yasya sānantaprabhedānimittaratiḥ, tatsvarūpasya paramānandasyopabhogaḥ, taddvāreṇa}.
	Of the Tibetan translation, while \TVA\ renders all words found in the Sanskrit text as constitued in some form, it is hard to intrepret if one does not remove or modify various instrumental and genitive particles.
	From \emph{rūpavajrā°} up to \emph{sambhujyamānaṃ}, \TVA\ reads: \emph{gzugs rdo rje la sogs pa nas | sgrol ma'i mthar thug pa'i lha mo'i tshogs kyis mtshan ma med pa'i dga' ba'i rang gi ngo bo'i rab tu dbye ba dpag tu med pas mchog tu dga' ba la nye bar longs spyod pa'i sgo nas | gzugs brnyan dang 'dra bas yang dag par longs spyod pa}. \TVB\ has the same readings, apart from the two suspected lacunae mentioned above.
} pratibimbavat [\EDD\ p.\ 138] sambhujyamānaṃ karuṇāsaṃvalitodārarūpatayā sambhogakāyarūpam, nānādhimuktivineyajanaparipācanārtham\footnoteA{
	\EDD\ misreports \MS\ as reading \emph{paripāvanārtha}.
}
anekavidhaprātihāryadvāreṇa\footnoteB{
	anekavidhaprātihārya°] \MS\ \EDD ; rdzu 'phrul dang cho 'phrul rnam pa du ma \TVA\ \TVB\ (anekaṛddhiprātihārya°)
} nirmitānantakulāntarbhūtasaṃbuddhabodhisattvaspharaṇasaṃhārakāritvena\footnoteB{
	°bodhisattva°] \conj\ (\TVB : byang chub sems dpa'i); °bodhi° \MS\ \EDD ; byang chub sems dpa' la sogs pa'i \TVA\ (°bodhisattvādi°) 
} nirmāṇakāyātmakam, śūnyatākaruṇābhinnabodhicitta\footnoteB{
	°bodhicitta°] \EDD ; °bodhicittā° \MS
}\hspace{0em}svabhāvāmalaprajñopāyasamādhisambhūtasatsukhāpūrṇam\footnoteA{
	See Sahajavilāsa, \emph{Svādhiṣṭhānakurukullāsādhana} (SāMā no.\ 183, p.\ 383): \emph{tataḥ prajñopāyāmalasamādhisambhūtasatsukhāpūrṇam iva svadehaṃ trailokya ca paśyet}.
} āsaṃsārasthitidharmaṃ\footnoteB{
	°dharmaṃ] \conj\ (\TIB : chos can); °dharmāṇāṃ \MS\ \EDD	
} apratiṣṭhitanirvāṇarūpaṃ nirmalanivātaniścalapradīpaśikhāprabandhanityatayā nirodhaśūnyaṃ caturthaṃ\footnoteB{
	caturthaṃ] \EDD ; caturtha \MS
} sādhyam.

\subsubsection{mantranaye pañcamaṃ sādhyam}
\begin{quote}
	kṛtvā sākṣāt svādhipaṃ sātarūpaṃ \\
	tyaktvopekṣājñānamātraṃ\footnoteB{
		tyaktvopekṣā°] \MS\ (\EDD\ \emph{incorrectly reports as \emph{tyajyo°}}) (TaRaa-Vi: tyaktvā, upekṣārūpaṃ yaj jñānaṃ); bhāvopekṣā° \EDD\ (\emd); no reflex in \TM\ CHECK
	} phalaṃ syāt |\\
	āsaṃsārasthāyi sattvārthakāri \\
	cintā\footnoteB{
		cintā°] \MS\PCreading\ \EDD ; cittā° \MS\ACreading
	}ratnaprakhyam\footnoteB{
		°prakhyam] \EDD ; °prakhyaṃm \MS
	} ekāntaśāntam || 12 ||\footnoteA{
		This verse is in Śālinī metre.
	}
\end{quote}

\noindent kṛtvetyādi.
sākṣāt svādhipaṃ kṛtvā, paścāt\footnoteB{
	paścāt] \EDD ; paścāta \MS
} tyaktvā, upekṣārūpaṃ yaj jñānaṃ tanmātraṃ sādhyaṃ syāt.
anyat sugamam.\footnoteB{
	sugamam] \EDD ; sūgamaṃ \MS
}
etad uktaṃ bhavati\emdash maṇḍalacakrarūpaṃ sākṣāt kṛtvā, paścāt tan nirodhya, upekṣājñānamātraṃ sādhyaṃ syāt pañcamam.

\subsubsection{mantranaye ṣaṣṭhaṃ sādhyam}
\begin{quote}
	kṛtvā sākṣān maṇḍalaṃ sātarūpaṃ \\
	paścāt tasya svecchayā nirvṛtiś\footnoteB{
		nirvṛtiś] \MS ; nirvṛtiṃ \EDD 
	}\footnoteA{
		Here the intended meaning, as stated in the commentary, is `cesation'. In the lengthy discussion of this position the author later uses the more expected term, `\emph{nivṛtti}', which is not metrically viable here.
	} ca |\\
	sattvārthasyāpy asty abhāvo na vāsmin \\
	prādurbhāvo nirvṛtād\footnoteB{
		nirvṛtād] \EDD ; nivṛtād \MS
	} asti yasmāt || 13 ||\footnoteA{
		This verse is in Śālinī metre.
	}
\end{quote}

% § 6.6_1
\noindent kṛtvetyādi.
kṛtvā sākṣān maṇḍalaṃ sātasaṃvalitam,\footnoteB{
	sātasaṃvalitam] \emd\ (\TIB : bde ba'i rang bzhin can); sātaṃ saṃvalitaṃ \MS\ \EDD
}
tasya sve\-ccha\-yā nirvṛtir nirodhaḥ.

% § 6.6_2
nanu yadi sākṣāt kṛtvāpi paścāt sve\-ccha\-yā nirodhayita\-[\MS\ fol.\ 5v]\-vyam,\footnoteB{
	nirodhayitavyam] \emd ; nirodhayitavyaḥ \MS\ \EDD
} tadā karuṇāyā anekakālābhyastāyā abhāvaḥ syāt.
tasyāś cābhāvāt sattvārthābhāvaḥ [\EDD\ p.\ 139] syād ity āśaṅkyāha\emdash sattvārthasyāpy asty abhāvo na vetyādi.
asmin pakṣe sattvārthābhāvo nāsti, yasmān nirvṛtāc cakrāt karuṇāsaṃvalitāt sattvārthasya prādurbhāvo 'sti.\footnoteA{
	The syntax of \TIB\ suggests reading \emph{karuṇāsaṃvalitasya}: \emph{'gags pa'i 'khor lo las snying rje'i rang bzhin can sems can gyi don} (\emph{'gags pa'i}] \TVB ; \emph{'gog pa'i} \TVA)
	However, \emph{karuṇāsaṃvalita} naturally qualifies \emph{cakra} and not \emph{sattvārtha}.
}

% § 6.6_3
etenaitad evāha\emdash sātasaṃpūrṇacakraṃ\footnoteA{
	Here one may wish to emend to \emph{sātasaṃpūrṇaṃ cakraṃ} to avoid the \emph{karmadhāraya}, given that the author did not previously use a \emph{karmadhāraya} when referring to this (e.g., \emph{maṇḍalaṃ sātasaṃvalitam}).
	Nevertheless, such a \emph{karmadhāraya} is not in any obvious way out of the normal scope of Vāgīśvarakīrti's usage.
} sākṣāt kṛtvā, yāvad iṣṭaṃ kālaṃ vyavasthāpya, paścāt tasya sarvathaiva pradīpavan nirodhaṃ kṛtvā sthātavyam.\footnoteB{
	nirodhaṃ kṛtvā sthātavyam] \MS\ \EDD ; 'gog pa yin no \TIB\ (nirodhaḥ)
}\footnoteA{
	It is possible that \emph{kṛtvā sthātavyam} was missing from the original text or from the version of it consulted by the Tibetan translators; alternatively, it is possible that the translators simply didn't feel it was necessary to explicitly render.
	The agent of \emph{sthātavya} can be understood to be the unspecified \emph{sādhaka} who is also the agent of the gerunds earlier in the sentence.
	Although a genudive of the causative of \emph{√sthā}, \emph{sthātavya} here has no object that is specified apart from the \emph{sādhaka} himself: i.e., he should make himself rest or establish himself in a state by doing what is described.
	The construction is frequently used in the \emph{Hevajratantra}, such as in 2.3.44: \emph{satataṃ devatāmūrtyā sthātavyaṃ yoginā yataḥ}; `\ldots\ for the \emph{yogin} should always establish himself/remain with the form of the deity.'
}
yadā punaḥ sattvārthābhilāṣo bhavati, tadā\footnoteB{
	tadā] \MS\ \EDD\ \TVB\ (de'i tshe); de'i \TVA\ (tad° ?)
} niruddhād eva cakrāntaram utpādya sattvārthaḥ kartavyaḥ.
cakrāntarotpāde\footnoteB{
	cakrāntarotpāde] \EDD ; cakrāntaropāde \MS
} 'pi ciraniruddhād\footnoteB{
	ciraniruddhād] \emd\ (\TIB : ring du 'gags pa'i); citaniruddhād \MS ; cittaniruddhād \EDD
} eva cakrād yathābhavyatayā\footnoteB{
	yathābhavyatayā] \emph{variant word division in} \EDD : yathā bhavyatayā
} vineyānāṃ yathābhilaṣitaprāptir bhavatīti ṣaṣṭham.\footnoteB{
	ṣaṣṭham] \MS\ (ṣaṣṭhaṃ) \EDD\ \TVB\ (drug pa'o); bsgrub par bya ba drug pa'o \TVA\ (sādhyaṃ ṣaṣṭham)
}

% § 6.7
\subsubsection{mantranaye saptamaṃ sādhyam}
\begin{quote}
	kṛtvā sphuṭaṃ rūpam abhīṣṭam eṣāṃ \\
	paścān nirodhaṃ\footnoteB{
		nirodhaḥ] \emd ; nirodha(ṃ) \MS\ (\emph{fort.\ corr.} ḥ); nirodhaṃ \EDD
	}\footnoteA{
		Both readings—\emph{nirodhaḥ} and \emph{nirodhaṃ}—are possible, but the former is supported by the following two verses, which have a similar structure in the second \emph{pāda} with a nominative form preceding \emph{āha}: \emph{°svādas turyaṃ sekam āhāvaraṃ tat |}
	} phalam āha kaścit |\\
	abhinnarūpaś ca yato nirodho \\
	na pakṣabhede 'pi tato 'sti bhedaḥ || 14 ||\footnoteA{
		This verse is in Rāmā metre.
	}
	% Rāmā, GGLGGLLGLGG X2, LGLGGLLGLGG X2
\end{quote}

% § 6.7_1
\noindent kṛtvetyādi.
ṣaṇṇāṃ pakṣāṇām anyatamasya phalasya\footnoteB{
	anyatamasya phalasya] \conj ; arthaphalasya \MS\ \EDD ; nang nas 'bras bu \TIB
} sādhyatvād yad yad evābhiṣṭaṃ\footnoteB{
	phalasya sādhyatvād yad yad evābhiṣṭaṃ] \MS\ \EDD ; 'bras bu bsgrub bya gang kho na \TVA\ (phalaṃ sādhyaṃ yad eva); 'bras bu bsgrub bya gang kho na mngon par 'dod pa \TVB\ (phalaṃ sādhyaṃ yad evābhiṣṭaṃ)
} tad\footnoteB{
	tad] \EDD\ \TVB\ (de); sad \MS ; \emph{no reflex in} \TVA
} eva sākṣāt kṛtvā, paścāt sarvathaiva pradīpavan nirodha uttarakālaṃ sattvārthādiśūnyaḥ sākṣāt kartavyaḥ.

% § 6.7_2
nanu ṣaṭpakṣabhedena ṣaḍ eva\footnoteB{
	ṣaḍ eva] \EDD ; ṣatreva \MS
} nirodhāḥ syuḥ. tat katham eka eva nirodha ity āśaṅkyāha\emdash abhinnetyādi. abhinnaṃ\footnoteB{
	abhinnaṃ] \EDD ; abhinna \MS
} rūpaṃ yasya sa tathā.\footnoteB{
	sa tathā] \emd ; tat tathā \MS\ \EDD
} na hi nirodhānāṃ ṣaṭpakṣalakṣaṇabhede 'pi bhedo 'sti, abhāvaikarūpatayā nirodhasya samānatvāt.
ayam arthaḥ\emdash anyatamapakṣaṃ sākṣāt kṛtvā paścāt tasya santānocchedarūpo nirodha iti saptamaṃ sādhyam.

% § 7
\subsection{caturthaseke vipratipattayaḥ}
\subsubsection{caturthaseke vipratipattiḥ prathamā}
\begin{quote}
	prajñājñānād uttaraṃ bodhicittā-\\
	svādas turyaṃ sekam\footnoteB{
		sekam] \EDD ; seṣam \MS
	} āhāvaraṃ tat |\\
	yasmāt\footnoteB{
		yasmāt] \EDD\ (\TIB : gang phyir) (TaRaA-Vi: yasmāt); paścāt \MS
	} sarvo bhāvanāsu prayāso \\
	vyarthaḥ prāptas tatphalasya prasiddheḥ\footnoteB{
		prasiddheḥ] \MS\ \EDD ; rab tu mi rung phyir \TM\ (aprasiddheḥ?)
	}\footnoteA{
		\TM 's reading \emph{rab tu mi rung phyir} is suprising, given that the commentary, presumably executed by the same tranlator, reads \emph{rab tu grub pa nyid [kyi phyir]}.
	} || 15 ||\footnoteA{
		This verse is in Śālinī metre.
	}
\end{quote}

% § 7_1
\noindent [\EDD\ p.\ 140] prajñājñānetyādi.
prajñājñānopadeśād uttarakālaṃ\footnoteB{
	prajñājñānopadeśād uttarakālaṃ] \MS\ \EDD ; shes rab dang ye shes ni shes rab ye shes te | dbang bskur ba'i bye brag go || phyis ni 'das pa'i 'og tu'o || gang zhe na | \TVA\ (prajñājñānetyādi. prajñā ca jñānaṃ prajñājñānaṃ sekaviśeṣaḥ. uttaram paścāt. kim iti); shes rab dang ye shes te | dbang bskur ba'i bye brag go || phyis te rdzogs pa'i dus kyi byang chub gang zhe na | \TVB
}\footnoteA{
	\TIB\ indicates that the text may have included a compound analaysis of \emph{prajñājñāna}, but if so, it is unclear what kind of compound this analysis signifies.
	If it is for a \emph{karmadhāraya}, we would expect the \TIB\ to read as it does elsewhere for such analyses, with something like \emph{shes rab kyang de nyid yin la | ye shes kang de nyid yin} (cf.\ the commentary on 5cd).
	The reading in \TVB\ is probably corrupt after \emph{phyis te}: \emph{byang chub} appears to have been moved from the following clause with \emph{bodhicitta} to this clause. 
	Perhaps the text should read \emph{phyis te rdzogs pa'i dus so || gang zhe na |}.
	Taken altogether, \TIB\ suggests the translator may have had a different reading here, but no compelling emendation is indicated.
} yat bodhicittasyāmṛtarūpasya\footnoteB{
	bodhicittasyāmṛtarūpasya] \emd\ (\TVA : byang chub kyi sems te bdud rtsi'i ngo bo); saṃ bodhicittasyāmṛtarūpasya \MS\ \EDD ; sems te bdud rtsi'i ngo bo \TVB\ (cittasya)
} rasanayā grahaṇam, tat turyaṃ caturthaṃ [\MS\ fol.\ 6r] sekam āha kaścit.
tac cāvaraṃ hīnam, vinikṛṣṭam iti yāvat.
kasmād avaram?
yasmāt sarvaprayāso mantramudrādevatādyākārabhāvanāsu punaḥ punar anuṣṭhānalakṣaṇas tathāgatokto\footnoteB{
	tathāgatokto] \EDD ; tathāgatoktau \MS
} vyarthaḥ prāptaḥ.\footnoteA{
	\TIB\ reflects basically the same words as transmitted in \MS\ but with an understnading that may reflect a different underlying reading.
	Whereas the Sanskrit text as transmitted in \MS\ appears to suggest primarily one thing that would be \emph{vyartha} on this position—namely, \emph{sarvaprayāsa} taught by the \emph{tathāgata}s that is characterised by repeated \emph{anuṣṭhāna} directed at meditations on mantras and so forth.
	\TIB , on the other hand, seems to understand two items that would be \emph{vyartha}: namely, \emph{sarvaprayāsa} and \emph{sgrub pa'i mtshan nyid}, probably \emph{anuṣṭhānalakṣaṇa}: \emph{gang gi phyir sngags dang phyag rgya dang | lha nyid la sogs pa'i rnam pa bsgom pa la yang dang yang du 'bad pa dang | gzhan yang de bzhin gshegs pas gsungs pa'i sgrub pa'i mtshan nyid don med pa thob par 'gyur ro ||} `Because it would follow that repeated effort in meditation \ldots\ and, what's more (\emph{gzhan yang}; Skt.\ \emph{ca}?), what is characterised as practice taught by the \emph{tathāgata}s would be useless'.
	This understanding is made more noteworthy by the manuscripts reading the dual \emph{tathāgatoktau}, but that may be just conicidental given the understnading reflected in \TIB\ is not very compelling on the level of overall sense.
}
kutaḥ?
tatphalasya bhāvanāsādhyasya phalasya bodhicittāsvādakāla eva prasiddhatvāt prāptatvāt,\footnoteB{
	prasiddhatvāt prāptatvāt] \MS\ \EDD\ \TVB\ (grub pa nyid dang | thob pa nyid dang |); rab tu grub pa nyid dang | \TVA\ (prasiddhatvāt)
} anyasya viśiṣṭasya phalasyābhāvād iti yāvat.

% § 7.2
\subsubsection{caturthaseke vipratipattir dvitīyā}
\begin{quote}
	prajñājñānād uttaraṃ prāptarāmā-\\
	svādas turyaṃ sekam āhādhamaṃ tat |\\
	yasmāt sarvo bhāvanādau prayatno \\
	buddhoddiṣṭo niṣphalaḥ saṃprasaktaḥ || 16 ||\footnoteA{
		This verse is in Śālinī metre.
	}
\end{quote}

% § 7.2
\noindent prajñetyādi.
prajñājñānād uttarakālaṃ yāḥ prāptā yathāmilitā rāmāḥ striyas tāsāṃ samāpattidvāreṇa\footnoteB{
	samāpattidvāreṇa] \EDD ; rig pa'i sgo nas \TVA\ (rig \emph{fort.\ pro} reg); reg pa'i sgo nas \TVB\ (sparṣadvāreṇa)
} ya āsvādaḥ, tat turyaṃ sekam.
tad apy adhamam.
śeṣaṃ gatārtham.

% § 7.3
\subsubsection{āgamasya vyākhyānam}
% § 7.3_1
atha\footnoteB{
	atha] \MS\ \EDD\ \TVB\ (de la); \emph{no reflex in} \TIB
} caturthaṃ tat punas tatheti\footnoteB{
	punas tatheti] \EDD\ (\emd); punar iti \MS
}\footnoteA{
	\emph{Samājottara} 113f
} vyākhyāyate. caturtham iti\footnoteB{
	caturtham iti] \MS\ \EDD\ \TVA\ (bzhi pa ni); \emph{no reflex in} \TVB
} prajñājñānaṃ tṛtīyam apekṣya caturtham ity ucyate.
tad iti tacchabdena tad eva prajñājñānaṃ tadrūpaṃ\footnoteB{
	tad eva prajñājñānaṃ tadrūpaṃ] \MS\ \EDD\ \TVB\ (shes rab ye shes de nyid kyi ngo bo de); shes rab ye shes kyi ngo bo de \TVA
} parāmṛśyate.\footnoteA{
	The referent of \emph{tat} in \emph{tadrūpaṃ} is evidently \emph{caturthaṃ}.
	\TVB\ essentially reflects the transmitted Sanskrit reading but \emph{eva}, if it is rendered by \emph{nyid}, is slightly out of place.
}
punar iti punaḥśabdena tasmād viśeṣaḥ.
viśeṣaś cātra\footnoteB{
	°ātra] \MD\ \EDD ; \emph{no reflex in} \TIB
} nirāsravaniruttarātyantasphītāvicchinnaprabandhapravāhitva\footnoteB{
	°niruttarātyantasphītāvicchinnaprabandhapravāhitva°] \MS ;   °nirantarātyantasphītāvicchinnaprabandhapravāhitva° \EDD\ (\emd); shin tu rgyas pa nyid dang | bar chad med pa nyid dang | rgyun mi 'chad par skye ba nyid kyi \TVA\ (°ātyantasphītatvanairantaryāvicchinnaprabandhapravāhitva°); shin tu rgyas pa nyid rgyun mi chad par skye ba nyid kyi \TVB\ (°ātyantasphītatvāvichinnaprabandhapravāhitva°)
}lakṣaṇaḥ.\footnoteB{
	°lakṣaṇaḥ] \EDD ; °lakṣaṇaṃ \MS
}\footnoteA{
	\EDD\ emends \emph{niruttara} to \emph{nirantara} stating that this reading is \emph{bhoṭānusārī}, but the situation in \TIB\ is slightly more complex.
	The transmitted Sanskrit suggests reading a string of adjectives starting with \emph{anāsrava} that qualify \emph{pravāhitva}.
	Here reading \emph{nirantara} and \emph{avicchinaprabandha} would lead to redundancy.
	\TIB\ instead renders a series of abstract nouns before \emph{pravāhitva}, with \TVA\ including something reflective of \emph{nairantarya} (\emph{bar ma chad pa nyid}).
	Both versions of \TIB\ lack a reflex of \emph{niruttara}.
} tatheti tathāśabdena tādṛśatvam abhidhīyate.
tādṛśatvaṃ ca yādṛśyā prajñādiyuktayā\footnoteB{
	°yuktayā] \conj\ (\TIB : dang ldan pa'i); °yuktyā \MS\ \EDD
} sāmagryā yādṛśaṃ prajñājñānam utpannam, paścād api tādṛśyaiva sāmagryā tathaiva cotpadyate, nānyatheti tathāśabdārthaḥ.

% § 7.3_2
atra ca lakṣyalakṣaṇabhāvenārtho\footnoteB{
	°ārtho] \MS\ \EDD\ \TVB\ (don); \emph{no reflex in} \TVA
}\footnoteA{
	For \emph{lakṣyalakṣaṇabhāva}, \TVA\ reads \emph{mtshon par bya ba'i don mtshan par byed pa'i dngos po}, which looks like a corruption for \TVB 's \emph{mtshon par bya ba dang} rather than anything indicative of a variant reading in the Sanskrit. 
} boddhavyaḥ.
lakṣyate 'neneti lakṣaṇam anubhūyamānaṃ prajñājñānam, apratīyamānasya lakṣaṇatvāyogāt, nāgṛhītaviśeṣaṇā [\EDD\ p.\ 141] viśeṣyabuddhir iti nyāyāt.
lakṣyate jñāyate pratipādyata\footnoteB{
	pratipādyate] \MS\ \EDD\ (pratipādyate); go bar bya zhing bsgrub par bya bas na \TVA ; khong du chud par byed bsgrub par byed pas na \TVB\ (pratīyate pratipādyata)
} iti\footnoteB{
	iti] \conj ; anenti \MS\ \EDD
}\footnoteA{
	Although \MS\ reads \emph{anenti}, and \TVB\ also reflects this with \emph{'dis}, by normal conventions the \emph{anena} here would indicate that the word being glossed, \emph{lakṣyaṃ} in this case, denotes the agent of verbal root, and this is clearly not the case.
	While the pronoun can potentially refer back to \emph{prajñājñāna}, it is also an easy scribal slip.
	The pronoun is not reflect in \TVA . 
} lakṣyaṃ sākṣāt kariṣyamāṇaṃ caturtham.

% § 7.4_1
\subsubsection{caturthaseke vipratipattis tṛtīyā}
atra caturthaṃ\footnoteB{
	caturthaṃ] \MS\ \EDD\ \TVB\ (bzhi pa ni); dbang ni rnam pa gsum dag tu | gyud 'di las ni rab tu grags || zhes gsungs pas na | bzhi pa \TVA\ (abhiṣekaṃ tridhā bhedam asmin tantre prakalpitam | iti vacanāc caturthaṃ)
} nāstīty eke.\footnoteA{
	\TVA\ adds near the beginning of this sentence \emph{Samājottara} 113ab: \emph{abhiṣekaṃ tridhā bhedam asmin tantre prakalpitam |}.
} nanu caturtham ity etad asti tatpadam.\footnoteB{
	nanu caturtham ity etad asti tatpadam] \MS\ (nanu caturtham ity etad asti | tat padan) \EDD ; de ltar de bzhin bzhi pa yang || zhes bya ba'i tshig bcom ldan 'das kyis gsungs pa yod pa ma yin nam | \TVA\ (nanu caturthaṃ tat punas tatheti padaṃ bhagavatoktam); de lta na de ma yin pa gzhan de ltar de bzhin bzhi pa yang zhes bya ba der bzhi pa zhes bya ba'i tshig bcom ldan 'das kyis gsungs pa yod pa ma yin nam | \TVB\ (nanu anyatra [? - de ma yin pa gzhan] caturthaṃ tat punas tathety asmin [? - der] caturtham iti padaṃ bhagavatoktam)
}\footnoteA{
	There is little doubt about the meaning of the text here, but its constitution is not very secure.
	Both Tibetan translation suggest that the whole of \emph{Samājottara} missing 113f was cited.
	While \TVA\ offers a somewhat cleaner text, \TVB\ again may reflect something closer to \MS , with a pronoun immediately following \emph{iti} and the word `\emph{caturtha}' marked off by an \emph{iti} on its own.
	Various proposals could be entertained for a smoother Sanskrit text, but what \MS\ transmits can be understood: `[Objections]: But there exists (\emph{asti}) a word (\emph{pada}) for that (\emph{tat}) [fourth initiation]—namely, this (\emph{etat}): {`}``the fourth [is that again like that]''.'
} tat kathaṃ nāstīty ucyate? satyam, upadeśasaṃrakṣārthaṃ\footnoteA{
	After rendering \emph{upadeśasaṃrakṣārthaṃ} (\emph{man ngag bsrung bar bya ba'i phyir dang}), \TVA\ has apparently suffered from an eyeskip and resumes with its translation of \emph{pratipādayiṣyamāṇatvāc ceti}.
} sattvavyāmohanāya ca tṛtīyam eva caturthaśabde[\MS\ fol.\ 6v]noktaṃ bhagavatā. anyathā tat punar\footnoteB{
	tat punar] \MS\ \EDD ; \TVA : \emph{not available}; de ltar de bzhin bzhi pa yang \TVB\ (caturthaṃ tat punas tathā)
}\footnoteA{
	Here \TVB\ quotes again the entire \emph{pāda} of \emph{Samājottara} 113f.
	This is an undesirable reading: It is specifically the words \emph{tat punaḥ} that indicate the Buddha real intention of speaking of \emph{caturtha}, not the entire \emph{pāda}.
} iti noktaṃ syāt.

% § 7.4_2
tad atyantāsaṃgatam, caturthasya pramāṇasiddhasya pratipāditatvāt pratipādayiṣyamāṇatvāc ceti.\footnoteA{
	Tib.\ discusses two further \emph{pakṣa}s here: that the fourth referred to in \emph{Samājottara} 113f is the four \emph{aṅga} of \emph{sevā} and so forth; and what appears to be the idea that the fourth initiation consists in the third accompanied by its fruits (\emph{'bras bu dang bcas pa}).
	Of these the first is rejected on account of its rendering cultivation meaningless, and the latter is rejected as \emph{siddhasādhana}.
}

% § 7.5
\subsubsection{lakṣyasya vicāraṇam}
atra lakṣaṇaṃ prajñājñānaṃ pratītam eva sarvaiḥ. lakṣye\footnoteB{
	lakṣye] \EDD\ (\emd) \TIB\ (mtshon par bya ba la); lakṣyā \MS
} paraṃ vyāmohaḥ. tad vicāryate. lakṣyaṃ hi bhaved\footnoteB{
	lakṣyaṃ hi bhavet] \conj\ (\TIB : mtshon par bya ba yang srid na); lakṣyaṃ hi bhagavat \MS\ \EDD\ (°gavad)
} artharūpaṃ vā syāt jñānarūpaṃ vā. na tāvad\footnoteB{
	tāvad] \MS\ \EDD ; \emph{no reflex in} \TIB
} artharūpam, arthasyaivābhāvāt,\footnoteB{
	arthasyaivābhāvāt] \conj\ (\TIB : don nyid med pa'i phyir); arthasyaikasyābhāvāt \MS\ \EDD
}\footnoteA{
	\MS 's reading \emph{arthasyaikasya} is problematic. At face value, it would mean `a single external object', whereas the argument is clearly concerning all external objects. Even if the meaning of \emph{eka} were strained and taken in the sense of `unitary', the following reason would become tautological.
	Emending \emph{ekasya} to \emph{eva} is also compelling based on the \TIB , which clearly reflects an \emph{eva} with \emph{nyid}.
} ekānekaviyogitvena pramāṇena tasya nirākṛtatvāt. mantranaye ca vijñānavādamadhyamakamatayor\footnoteB{
	°matayor] \EDD\ \TIB\ ('dod gzhung); °tamayor \MS
} eva pradhānatvād\footnoteA{
	\TVA\ finishes the clause ending \emph{pradhānatvāt} with a \emph{rdzogs tshig}: \emph{gtso bo nyid yin pa'i phyir ro}.
	This creates an incomplete sentence with the clause pointing neither forwards nor backwards, since the clause ending in \emph{nirākṛtatvāt} also ends with \emph{phyir ro}. 
	The translation may be corrupt, or perhaps the translator was uncertain about how to construe the clause.
	Here the reason should probably point forward: although it does support the non-primacy of external objects, enough reasons have been given to support their general non-existence, and the primacy of awareness in the Vijñānavāda and Madhyamaka positions can be seen as a basis on which the \emph{lakṣya} could be accepted as \emph{jñāna}.
} jñānarūpaṃ vā syāt. jñānaṃ ca sākāraṃ vā nirākāraṃ vā. sākāram api citrādvaitarūpaṃ\footnoteA{
	Here and at the end of the next paragraph, \TIB\ renders \emph{citrādvaita} as \emph{shes pa gnyis med}, as if reading \emph{cittādvaita}.
	The more expected rendering is \emph{sna tshogs gnyis med}.
} vā syād anekarūpaṃ vā syād iti vikalpāḥ.

% § 7.5.1
\subsubsubsection{sākārasya vijñānasya nirākaraṇam}
tatra sākāravijñānaṃ sarvathaiva gagaṇakamalavan nāstīti nirākāravādino bruvate. nanu nīlapītaśuklādighaṭapaṭaśakaṭādi\footnoteB{
	°śakaṭādi°] \EDD\ (\emd) \TIB\ (shing rta); °prakaṭādi° \MS
}rūpe\-ṇā\-kā\-rāḥ\footnoteB{
	°ākārāḥ] \conj ; ((cā))kārāḥ \MS ; vākārāḥ \EDD
} pratibhāsante\footnoteB{
	pratibhāsante] \EDD ; pratibhāṣante \MS
} pratyakṣataḥ. te cārthasyābhāvād jñānarūpā eva. tat kathaṃ sākāraṃ nāstīti?\footnoteA{
	\TVB\ presents this argument differently than what is transmitted in Sanskrit but maintains logical flow: \emph{don (rnam pa) de dag kyang med pa'i phyir shes pa'i ngo bo nyid kyang med yin na | de ji ltar rnam pa dang bcas pa ma yin zhe na |} `Because those objects [i.e., \emph{ākāra}s] also do not exist, the nature of cognition too cannot exist. So how can cognition not have \emph{ākāra}s?'
	\TVA 's formulation is unclear: \emph{don de dag la med pa'i phyir shes pa'i ngo bo nyid yin na | de ji ltar rnam pa dang bcas pa ma yin zhe na |}
} satyam. pratibhāsanta evākārāḥ, param alīkarūpeṇa.\footnoteB{
	alīkarūpeṇa] \MS\ \EDD\ \TVB\ (brdzun pa'i ngo bor); brdzun pa yin no \TVA\ (alīkam)%
} alīkarūpatā\footnoteB{
	alīkarūpatā] \MS\ \EDD\ \TVB\ (brdzun pa'i ngo bo nyid); brdzun pa nyid \TVA\ (alīkatā)%
} caikānekaviyogitvena pramāṇalakṣaṇena\footnoteB{
	°viyogitvena pramāṇalakṣaṇena] \emd ; °viyogitvapramāṇalakṣaṇena \MS\ \EDD ; dang bral ba'i mtshan nyid kyis \TVA\ (°viyogalakṣaṇena); dang bral ba'i tshad ma'i mtshan nyid kyis \TVB\ (°viyogapramāṇalakṣaṇena)
} prasiddhā. tasya ca pramāṇasvarūpasyānyatra\footnoteB{
	pramāṇasvarūpasyā°] \EDD ; pramāṇa((pe))rūpasyā° \MS
} kathitatvān, neha\footnoteB{
	neha] \EDD ; eha \MS
} pratanyate. alīkatvaprasiddhā ca māyāmayā ivākārā bhrāntirūpāḥ prakāśante.\footnoteB{
	prakāśante] \MS\ (prakāsante); prakāśyante \EDD
} bhrāntinivṛttau ca nirākāram eva\footnoteB{
	nirākāram eva] \MS\ \EDD\ \TVB\ (rnam pa med pa kho na); rnam pa med pa de kho na \TVA\ (nirākāram eva tad)
} śuddhasphaṭikasaṃkāśaṃ pāramārthikaṃ\footnoteB{
	pāramārthikaṃ] \EDD\ (\emd); pārarthikaṃ \MS
} siddhaṃ bhavati.\footnoteB{
	bhavati] \MS ; bhavatīti \EDD
} ataś citrādvaitarūpam anekarūpaṃ ca sākāraṃ vijñānam astīti vikalpadvayaṃ nirastaṃ bhavatīti.

% § 7.5.2_1
\subsubsubsection{nirākārasya vijñānasya samarthanam}
nanu nirākāram api vijñānam upalabdhilakṣaṇaprāptaṃ svapne 'pi nopalabhyate. tat kathaṃ tad asti paramārthata\footnoteB{
	paramārthata] \emd ; paramārtham \MS\ \EDD
} i[\MS\ fol.\ 7r]ty ucyate?
ucyate.\footnoteB{
	ucyate] \MS\ \EDD\ \TVB\ ('on te); \emph{no reflex in} \TVA
}\footnoteA{
	(TO EXPAND AND REORGANISE) Here \emph{'on te} in \TVB\ isn't a strong reflex of \emph{ucyate}, but like \emph{ucyate} it does explicitly mark a change in \emph{pakṣa}.
	The \emph{ucyate} ending the previous sentence may be suspect.
	A similar formulation was used in the previous paragraph: \emph{tat kathaṃ nāstīty ucyate}? There \TIB\ reads: \emph{ji ltar med ce na |} (\TVB); \emph{de ci ltar med ces brjod |} (\TVB).
	Here, for \emph{tat kathaṃ tad asti paramārthata ity ucyate}, \TIB\ reads: \emph{de ji ltar na don dam par grub par yod pa zhes bya zhe na} (\TVA); \emph{de ji ltar na don dam par yod par grub pa zhes bya |} (\TVB).
	From this it is difficult to draw firm conclusions, but \emph{ces brjod} and \emph{zhes bya} probably more strongly point towards \emph{ity ucyate} rather than simply \emph{iti}.
}
sukhākāraṃ vijñānam \footnoteB{
	antaḥ°] \MS\ \EDD\ \TVB\ (nang na); rnams ni \TVA\ (\emph{probably corruption})
}antaḥpa\-ri\-sphu\-ra\-drū\-paṃ nirākāraṃ saṃvedyata eva.\footnoteA{
	\TIB\ changes the subject of the sentence from \emph{vijñāna} to the \emph{ākāra}s of \emph{vijñāna}.
	\TVA\ is likely corrupt with \emph{shes pa rnams ni} in place of \TVB 's \emph{shes pa'i nang na}.
} nīlādyākārāḥ punar alīkāḥ pratibhāsante. % annotated ms. circles kā in nīlādyākārāḥ
anyathā teṣāṃ satyatve sarva evākārāḥ satyāḥ syuḥ.
tathā hi grāhyagrāhakabhāvādikam api satyaṃ [\EDD\ p.\ 142] syāt.
tataś ca sarveṣām eva satyapratibhāsatvena muktiprasaṅgaḥ,\footnoteB{
	muktiprasaṅgaḥ] \emd\ (\TIB : grol ba nyid du thal bar 'gyur te); yuktiprasaṅgāt \MS ; muktiprasaṅgāt \EDD\ (\emd)
} keṣāñcid api mithyāpratibhāsasya bhrāntirūpasyāpratibhāsanāt.\footnoteA{
	Both Tibetan translation exhibit various degrees of corruption and/or confusion here: \emph{cung zhig kyang log pa'i rnam par ngo bo ni snang ba'i phyir ro ||} (\TVA) \emph{cung zhig dang log pa'i rnam par 'khrul pa'i ngo bo mi snang ba'i phyir ro ||} (\TVA).
	There is a possibility that \emph{log pa'i rnam pa} reflects \emph{mithyākāra} instead of \emph{mithyāpratibhāsa}.
} tathā coktam—

\begin{quote}
	draṣṭavyaṃ\footnoteB{
		draṣṭavyaṃ] \EDD ; draṣṭavya \MS
	} bhūtato bhūtaṃ bhūtadarśī vimucyate~|\footnoteA{
		\emph{Abhisamayālaṅkāra} 5.21; \emph{Ratnagotravighāga} 154; \emph{Pratītyasamputpādahṛdayakārikā} 7; etc.
	}
\end{quote}

\noindent tasmād akāmakenāpi nīlādyākārāṇām alīkatvam evaiṣṭavyam.
sukhādikaṃ nirākāraṃ\footnoteB{
	nirākāraṃ] \MS\ \EDD ; rnam pa \TVA\ (ākāraṃ); rnam pa brdzun pa \TVB\ (alīkākāraṃ)
} satyam upalabhyate.
tat kathaṃ nopalabhyata iti.

% § 7.5.2_2
nanu sukhādyākāram sākāram eva vijñānam\footnoteB{
	sākāram eva vijñānam] \conj ; eva vijñānam \MS\ \EDD ; shes pa yang rnam pa dang bcas pa kho na \TVA ; rnam pa dang bcas pa'i kho na shes pa \TVA\ (api sākāram eva jñānam) 
}\footnoteA{
	The word \emph{sākāram} appears to have been omitted from the text transmitted in \MS . It is supported by the Tibetan translations and can be inferred by the reason \emph{sukhāder ākārasvabhāvatvā}, and the response to this objection later in the paragraph.

	\hspace*{1em}\TIB\ may reflect also the inclusion of an \emph{api} somehwere in this sentence given the particle \emph{yang} (e.g., \emph{sukhādyākāram api}).
	Here again \TVB\ as altered the argument slightly: \emph{bde ba la sogs pa'i rnam pa yang rnam pa dang bcas pa'i shes pa kho na la dmigs pa yin te |}; `Forms such as pleasure too are perceived with only a cognition that has forms.'
	\TVA\ is closer to the Sanskrit syntax: \emph{bde ba la sogs pa'i rnam pa'i shes pa yang rnam pa dang bcas pa kho na la dmigs pa kho na yin te |}; `Cognition too that has the forms of pleausre and the like are only perceived to be none other than [cognitions] with forms.' 
} upalabhyate, sukhāder ākārasvabhāvatvāt.
na ca sukhādyākāraśūnyaṃ jñānaṃ\footnoteB{
	jñānaṃ] \MS\ \EDD ; rnam par shes pa \TIB\ (vijñānaṃ)
} svapne 'pi saṃvedyate.
sakalabhrāntivigamād aṣṭamyāṃ bhūmāv upalabdhilakṣaṇaprāptir bhavatīty atrāpi kośapānaṃ\footnoteB{
	kośapānaṃ] \MS\ (kosapānaṃ); śapathollaṅghanaṃ \EDD\ (\emd)
} vinā anyan na\footnoteB{
	anyan na] \EDD ; anyatra \MS
} pramāṇam asti prasādhakam iti.\footnoteB{
	iti] \MS\ \EDD\ \TVA\ (zhe na); \emph{no reflex in} \TVB
}
tad asat,\footnoteB{
	tad asat] \conj\ (\TIB : de ni bden pa ma yin te); tad \MS\ \EDD ; asad etat \possibleconj
} abhiprāyāparijñānāt, sukhādyākārasyaiva\footnoteB{
	sukhādyākārasyaiva] \MS\ \EDD ; bde ba la sogs pa nyid \TVA ; bde la sogs pa nyid \TVB\ (sukhāder eva)
} nīlādyākārarahitasya vijñānasya nirākāratveneṣṭatvāt.\footnoteB{
	nirākāratveneṣṭatvāt] \MS\ \EDD\ \TVB\ (rnam pa med pa nyid du 'dod pa nyid kyi phyir); med pa nyid du 'dod pa'i phyir \TVA
} tac cedānīm eva svasaṃvedanapramāṇasiddhaṃ sakalaprāṇabhṛtām astīti kathaṃ nopalabdhiḥ?
% Structure of argumnetation in TIB is hard to follow, but it's maybe not completely off. First pakṣa is marked with rhetorical question at the end of saṃvedyate. Second pakṣa is introduced in TVB with 'on te, but this kind of sounds like the siddhāntin speaking again, as it connects causally to the 'tad asat'. TVA marks off the portion beginning sakalabhrāntivigamād with a zhe na at the end, but a bit hard to see how it all connects. TVB also possible indicates a change of pakṣa for the final sentence. No reflex of nanu in the following paragraph.

% § 7.5.3_1
\subsubsubsection{madhyamakamatasya samarthanam}
nanu tad\footnoteB{
	nanu tad \MS\ \EDD ; tat \possibleconj
} apy ekānekasvabhāvaviyogād alīkam eva bhrāntimātram, ekānekasvabhāvarahitatvasya\footnoteB{
	°rahitatvasya] \emd\ (\TIB : dang bral ba nyid kyis); °rahitasya \MS\ \EDD
} sākāranirākāravijñānavyāpitvāt.\footnoteB{
	°vijñāna°] \MS\ \EDD ; shes pa \TIB\ (jñāna)
}

% § 7.5.3_2
nanv anena nyāyena sakalasākāranirākāravijñānasyā\footnoteB{
	°vijñānasyā°] \MS\ \EDD ; shes pa \TIB\ (°jñānasyā°)
}\-lī\-ka\-tva\-pra\-sā\-dha\-nān na kiñcid api pāramārthikaṃ vastutattvam asti.\footnoteB{
	asti] \conj ; astīti \MS\ \EDD\ (astīti?); \emph{no reflext in} \TIB)
}\footnoteB{
	The \emph{iti} following \emph{asti} in \MS\ is superfluous with \emph{tat} starting the next sentence in the sense of \emph{tasmāt}, continuing the objection.
} tat kathaṃ lakṣyasya svarūpaṃ pramāṇata upalakṣayitavyam?
naiṣa doṣaḥ, madhyamakamate pramāṇato 'līkatāsiddhāv api \footnoteB{
	māyopamapratibhāsamātrasyai°] \MS\ \EDD ; snang ba tsam dang sgyu ma lta bu \TVA\ (māyopamasya pratibhāsamātrasya cai°); snang ba sgyu ma lta bu ma \TVB\ (māyopamapratibhāsasyai°)
}māyo\-pa\-ma\-pra\-tibhāsamātrasyaikānekasvabhāvarahitasya dharmirūpasyāpratiṣedhāt.
tatraiva cālīke pratibhāsamātre lakṣyalakṣaṇasaṃsāranirvāṇa[\MS\ fol.\ 7v]maṇḍalacakrādibhāvanāsakalajaga\-da\-rtha\-kri\-yā\-dī\-nām\footnoteB{
	°bhāvanā°] \MS ; °bhāvanā \EDD\ (variant word division); bsgoms pas \TIB\ (bhāvanayā)
} avyāhatā vyavasthā\footnoteB{
	vyavasthā] \MS ; vyavasthā ca \EDD\ (\emd)
} sidhyati.\footnoteB{
	sidhyati] \conj\ (\TIB : grub pa yin no); sidhyatīti \MS\ \EDD
}\footnoteA{
	\EDD\ appears to understand the text as saying that both \emph{bhāvanā} and \emph{sakalajagadarthakriyādīnāṃ vyavasthā} are established.
	\TIB\ renders \emph{bhāvanā} in the third case, suggesting it may have been seen outside of the compound or seen within the compound but understood as having a \emph{tṛtīya} relationship with \emph{sakalajagadarthakriyā}.	
	We understand a compound beginning with \emph{lakṣyalakṣaṇa} up to \emph{sakalajagadarthakriyādīnām} providing a list of that for which the \emph{vyāvasthā} is still established in the Madhyamaka system.

	\hspace*{1em}Here too \MS\ seems to transmit a superfluous \emph{iti}, here following \emph{sidhyati}.
} tathā coktam\emdash 

\begin{quote}
	buddhatvaṃ vajrasattvatvaṃ saṃvṛtyaiva prasādhayet~|\footnoteA{
		\emph{Kurukullākalpa} 3.16cd
	}
\end{quote}

\noindent iti.\footnoteB{
	iti] \EDD ; deest \emph{in} \MS
}

% § 7.5.3_4
nanu sarvam eva vastujātam alīkarūpatayā niḥsāram, tadā kimarthaṃ maṇḍalacakrādibhāvanāprayāsaḥ\footnoteB{
	maṇḍala°] \MS\ \EDD ; bri ba'i 'dkyil 'khor (lekhyamaṇḍala°)
} kriyate?
asad etat,

\begin{quote}
	mithyādhyāropahānārthaṃ\footnoteB{
		mithyādhyāropahānārthaṃ] \emd ; mithyādhyāropaṇārthaṃ \MS\ \EDD
	} yatno 'saty api\footnoteB{
		'saty api] \MS ; 'styopi \EDD
	} [\EDD\ p.\ 143] bhoktari~|\footnoteB{
		moktari] \emd\ (\TVA : grol byed; \TVB grol ba po) ; bhoktarī° \MS\ (\emph{the letter} no \emph{is added abhove} bho); muktaye \EDD\ (\emd)
	}\footnoteA{
		\emph{Pramāṇavārttika}, Pramāṇasiddhi 192cd. Verse 192 is frequently cited in Buddhist and non-Buddhist texts alike and is transmitted with the readings \emph{bhoktari} and \emph{moktari} in the final \emph{pāda}, with the latter better represented in the core witnesses of and texts releated to the \emph{Pramāṇavārttika} (for some references see \cite[168]{pecchia2015}).
	}
\end{quote}

\noindent iti vacanāt. yady api vicāryamāṇaṃ pāramārthikaṃ vasturūpaṃ nāsti, tathāpy ahaṃ sukhī bhaveyaṃ mā\footnoteB{
	mā] \EDD\ (\emd); deest \emph{in} \MS
} duḥkhy abhūvam iti tṛṣṇā sakalaprāṇabhṛtām asti.
yathā tulye 'pi mithyātve śubhāśubhasvapnayoḥ śubhasvapnadarśanāt saumanasyam, aśubhasvapnadarśanāc ca daurmanasyam, tadapanayanāya ca saddharmapāṭhamantrajāpādau pravṛttir bhavati, tathā mithyātvāviśeṣe 'pi duḥkhādiprākṛtavikalpahānāya\footnoteA{
	cf.\ \emph{Samantabhadrasādhana} (as quoted in Kamalanātha's \emph{Ratnāvalī} ad HeTa 2.2.45, fol.\ 16r6): prākṛtavikalpavṛttair aparaṃ na hi kiñcad asti bhavaduḥkham | tasya viruddhaṃ caitat sākṣādavagamyate cetaḥ ||
} samyaksaṃbodhilakṣaṇaprāptaye\footnoteB{
	°lakṣaṇaprāptaye] \MS\ \EDD ; mtshan nyid kyi 'bras bu thob par bya ba'i phyir \TVA ; mtshan nyid 'bras bu thob par bya ba'i phyir \TVB\ (°lakṣaṇaphalaprāptaye)
} ca prekṣāvatām arthināṃ pravṛttir bhaviṣyatīti.

\subsection{saptavidheṣu sādhyeṣu sārāsāravicāraṇam}
nanu yadarthas tv ayam\footnoteB{
	yadarthas tv ayam] \conj ; yadarthasvā'yaṃ \MS ; yadarthatvād ayaṃ \EDD 
}\footnoteA{
	An alternative conjecture for whethere \MS\ reads \emph{yadarthasvā'yam} could be \emph{yadarthas tavāyam}, but we see no reflex of a \emph{tava} in the Tibetan translations: \emph{rtsom pa 'di'i don gang yin pa} (\TVA); \emph{gal te gang gi don du (bzhi pa bshad pa'i bshad pa'i dus) 'di brtsams pa'i} (\TVB).
} ārambhaḥ so 'rthaḥ pralayaṃ gataḥ. tathā hi lakṣyalakṣaṇacintātra prastutā. sā ca vismṛtā,\footnoteB{
	vimisṛtā] \MS\ \EDD ; gtam gzhan du thal bas brjod pa'i phyir \TIB
}\footnoteA{
	\TIB\ may suggest a different reading (which cannot easily be guessed at), or it may simply elaborate on what is found in the Sanskrit text: \emph{de yang gtam gzhan du thal bas brjod pa'i phyir | gang du song ba mi shes so zhe na |}; `And (\emph{yang}) because you have spoken (\emph{brjod pa}) by moving on to (\emph{thal bas}) other topics (\emph{gtam gzhan}), where that (\emph{de}) [main topic] has gone is not known.'
} kva gateti na jñāyate.\footnoteB{
	jñāyate] \MS\ \EDD ; shes so zhe na \TIB\ (jñāyata iti [cet])
}

na tu\footnoteB{
	na tu] \conj ; nanu \MS\ \EDD
} kṛtaiva sā saptabhir bhedaiḥ?

satyam, kintu guḍagorasanyāyena.\footnoteB{
	guḍagorasanyāyena] \MS\ \EDD\ \TVB\ (bu ram dang dar ba'i tshul gyis); bu ram dang mngar ba nyid kyi tshul gyis \TVA
}\footnoteB{
	cf.\ verse 267 of Pramāṇasiddhi chapter in Prajñākāragupta's \emph{Pramāṇavārttikabhāṣya}:
\emph{arthānarthakriyāśakto guḍagorasakārakaḥ |
sarvajño 'pi na sevyatvaṃ prayāty anupakārataḥ ||}; `Because he is not helpful, a creator of [a mixture of] \emph{guḍa} and \emph{gorasa}, capable of doing both harm and good, does not becomes an object of service/devotion, even if he is omniscient' (ed-s p.\ 37).
	On this Yamāri comments: \emph{bu ram dar ba byed pa po || zhes gsungs te | 'di rigs pa dang mi rigs pa 'dres pa la grags pa yin no ||}, '... this is known as ``mixing what is appropriate and not appropriate''' (Tōh.\ 4226 fol.\ 12v6–7).
	Sāṃkṛtyāyana records a marginal note in his manuscript on the term: `\emph{lohita (?) guḍakārakaḥ, guḍagomayakāraka ity apekṣyate}.'
	
	\hspace*{1em}The author of the \emph{Vādarahasya} uses the term as well:  \emph{atadrūpaparāvṛttanīlākārātanmātragrahaṇam iti vyavasthāyāṃ nāpi viṣayasārūpyaṃ, (tadabhāvān) kā hi paramārthasadalīkarūpayoḥ samānarūpatā nāmetyādi guḍagorasayor ekatākaraṇaṃ kvopayuktam | bādhakapratyayād dhi tadalīkatvaṃ kiṃ prāgāropya cintā kriyate śeṣaś ca doṣo 'bhimānasyaiva cintyatvādityādir ajatapratītiparāmarśād gataḥ |} (p.\ 73–74); `In the system where there is grasping to more than just a blue form that excludes what is not of that nature, there is not even similarity to the object (because of its absence [?]). For what indeed could be the so-called similarity between what is ultimately real and what is unreal? Given this and similar [arguments], how is the unification of \emph{guḍa} and \emph{gorasa} useful?'
	The context here appears to be a refutation of the view that conceptual cognitions include both a universal and a real object.

	\hspace*{1em}Although it is evidently not a widely reference `\emph{nyāya}', the general idea seems to be that these two substances represent the appropriate and the inapproriate (or the useful and the useless), and that they should not be mixed.
	Precisely what substances, then, \emph{guḍa} and \emph{gomaya} refer to are then difficult to determine, as molasses and milk seem like a harmless combination.
} tathā hi na jñāyate, kiṃ tat sāram asāraṃ veti.

ucyate.

% § 8.1
\subsubsection{prathamasyāsāratvam}
mantranayavihitakramābhāvāt samāpattibhāvanāvaiyarthyād\footnoteB{
	samāpatti°] \MS\ \EDD\ \TVB\ (snyom par 'jug pa); lha'i rnal 'byor gyi snyoms par 'jug pa'i \TVA\ (devatāyogasamāpatti°)
} yuktyabhāvāc\footnoteB{
	yuktyabhāvāc] \EDD ; yuktābhāvāc \MS
} ca prathamasya niḥsāratā.
tathā hi samagrasāmagrīkaṃ yat\footnoteB{
	yat] \MS\ \EDD ; 'bras bu gang yin pa \TIB\ (yat phalaṃ) 
} tad avaśyam eva bhavati.
anyathā samagrasāmagrīkam eva tan\footnoteB{
	samagrasāmagrīkam eva tan] \MS\ \EDD\ \TVB\ (tshogs pa dang tshogs can nyid du de); de'i tshogs pa \TVA\ (tasya sāmagrī)
} na bhavet.
sākṣātkaraṇāvasthāyāṃ samagrasāmagrīkaṃ tad vartate.
tad avaśyaṃ tena\footnoteB{
	tena \MS\ \EDD\ \TVB\ (de); de'i 'bras bu \TVA\ (tena phalena)
} bhavitavyam. sati ca bhavet\footnoteB{
	sati ca bhavet] \conj ; sati ca bhavane na \MS\ \EDD ; de ltar gyur pas \TVA ; de ltar gyur pa \TVB\ (evaṃsati)
} prathamasya hānir iti.

% § 8.2
\subsubsection{dvitīyasyāsāratvam}
śarīrādyākāraśūnyasya kevalasātarūpasyānupalabdher\footnoteB{
	°labdher] \EDD ; °bdher \MS
} na dvitīyasya sāratā.
tathā hi pramāṇaniścitaṃ prekṣāvatā bhāvanīyam, na yathākathañcit.
pramā[\MS\ fol.\ 8r]ṇena saṃvalitarūpam eva sarvadopalabhyate.\footnoteB{
	saṃvalitarūpam eva sarvado°] \MS\ \EDD ; grub pa kho na \TVA\ (siddham eva); grub pa'i ngo bo thams cad du \TVB\ (siddharūpaṃ sarvado°)
}
tad eva sarvajanānāṃ kamanīyatayā pratibhāsate.
tasmāt kevalasya rucyabhāvāc\footnoteB{
	rucyabhāvāc] \MS\ \EDD ; mi dmigs pa'i phyir dang | 'dod par bya ba ma yin pa'i phyir dang | \TVA ; ma dmigs pa'i phyir dang | 'dod pa med pa'i phyir dang | \TVB\ (anupalabdhe rucyābhāvāc)
} \footnoteB{
	cakrākārasaṃvalita°] \MS\ \EDD ; 'khor lo'i rang bzhin \TVA\ (cakrasvarūpa°); 'khor lo'i rnam pa'i rang bzhin \TVB\ (cakrākārasvarūpa°)
}cakrākārasaṃvalitasyānupalabdheḥ\footnoteB{
	°syānupalabdheḥ] \emd\ (\TVA : mi dmigs pa'i phyir dang) (\TVB : ma dmigs pa'i phyir); °syaupalabdheḥ \MS ; °syopalabdheḥ \EDD
} sākṣāt kartum aśakyatvāc\footnoteB{
	aśakyatvāc] \EDD\ (\emd); aśakyatāc \MS ; mi nus ba'i phyir dang | 'bad pa nyid mtshungs pa'i phyir \TIB\ (aśakyatvād yatnasyaiva tulyatvāc)
}\footnoteA{
	\TIB\ suggests reading: \emph{kevalasyānupalabdheḥ rucyabhāvāc cakrākārasaṃvalitasyānupalabdheḥ sākṣāt kartum aśakyatvāt}.
	The addition of \emph{anupalabdheḥ} after \emph{kevalasya} renders the flow of logic less smooth and makes \emph{sākṣāt kartum aśakyatvāc} superfluous.
	\TIB\ also adds the reason \emph{'bad pa mtshung pa'i phyir} (`becaue the effort is equal'), which is a fitting argument: although according to this system only bliss is meditated on and achieved, this actually requires the same amount of effort as the systems that include deity forms.
} ca dvitīyasya kalpanāmātrateti.\footnoteB{
	kalpanāmātrateti] \EDD\ (\emd); kalpanātrateti \MS
}

% § 8.3
\subsubsection{tṛtīyasyāsāratvam}
nirupadravabhūtārthasvabhāvatvena sātmībhūtasya tyaktum aśakyatvāt, saṃvalitarūpasya [\EDD\ p.\ 144] bhedābhāvāt, prayojanābhāvāc ca na tṛtīyasya kalyāṇabhāvaḥ.\footnoteB{
	na tṛtīyasya kalyāṇabhāvaḥ] \conj ; na tṛtīyakalyanībhāvaḥ \MS\PCreading ; na tṛtīyakalyānībhāvaḥ \MS\ACreading ; na tṛtīyaḥ kalpanābhāvaḥ \EDD
}\footnoteA{
	Where we conjecture \emph{na tṛtīyasya kalyāṇabhāvaḥ}, \TIB\ reads: \emph{gsum pa dge ba ma yin te}.
	The \emph{kalyāṇatā} in the following paragraph is rendered with \emph{legs pa}.
	There too an abstract noun with another noun in the genitive case is not reflected, but such syntax would in any case be less natural in Tibetan.
	The reading of \ED\ (either a silent emendation or a misreading of the manuscript), \emph{na tṛtīyaḥ kalpanābhāvaḥ}, gives some sense (`the third is not without conceptual construction'), and for this we must supply a masculine headword such as \emph{pakṣa}. 
	There are other options to emend \MS 's reading, such as \emph{na tṛtīyasya kalyāṇatā} or perhaps \emph{na tṛtīyasya kalpanābhāvam}.
	Note that \emph{kalyāṇatā} in the following paragraph was also copied in \MS\ with a dental \emph{na}.
} tathā hi sahopalambhena\footnoteB{
	sahopalambhena] \EDD ; saholaṃbhena \MS
} tādātmyasiddhāv ekasya parityāge 'parasyāvaśyaṃ parityāgaḥ, na vā kasyacid iti. 

\subsubsection{caturthasya sārāsāratvavicāraṇam}
prapañcatvena bahuprayāsatvād vicārāsahatvena bhrāntirūpatayāparamārtharūpatayā ca na tṛtīyāntapakṣasya\footnoteB{
	tṛtīyāntapakṣasya] \emd\ (\TVA : gsum pa'i tha' ma'i phyogs) (\TVB : gsum pa'i mtha' ma'i phyogs); tṛtīyāntaḥ | pakṣasya \MS ; tṛtīyapakṣasya \EDD 
} kalyāṇateti.\footnoteB{
	kalyāṇateti] \EDD ; kalyānateti \MS
}
atra kecid yuktiṃ varṇayanti.\footnoteA{
	\TVA\ conveys a different meaning here: \emph{de la 'ga' zhig las rigs pa cung zhig cig brjod par mi bya ste |}
	It is possible that this sentence is corrupt (especially the \emph{las} after \emph{'ga' zhig}).
} prapañcarūpatvābhāve\footnoteB{
	prapañcarūpatvābhāve] \MS\ \EDD ; spros pa'i ngo bo nyid du gyur \TIB\ (prapañcarūpatve)
} 'pi sūkṣmasya bindvādeḥ punaḥ punar bhāvanayā sākṣātkaraṇaṃ yāvat, prayāsas tāvat sarvatraiva bhāvyavastuni sambhavati.\footnoteA{
	\TIB\ differs substantially in the second part of this sentence: \emph{de srid du 'bad pas yang dang yang du bsgoms pa'i phyir dang | thams cad du bsgom par bya ba dngos po nyid du yod la |} (\TVA); \emph{de srid du 'bad pas yang dang yang du bsgom pa'i phyir thams cad du bsgom par bya ba'i dngos po yod do ||} (\TVB).
	Of these, the intention behind \TVA\ is hard to discern, but \TVB\ can be translated: `For that long, because one repeated meditates with effort, the object of meditation remains.'
	Here the meaning is not compelling and indicates corruption or mistranslation.
}
tad atra yadi prayāsabhayam, na kiñcid api bhāvanīyam.

prapañcarūpatvād iti cet, prapañcāprapañcayor bhāvanāvasthāyāṃ ko viśeṣaḥ?\footnoteB{
	ko viśeṣaḥ] \emd\ (\TIB : khyad par ci zhig yod |); ko viśeṣa iti cet \MS\ \EDD
} nanu\footnoteB{
	nanu] \conj\ (\TIB : 'on te); deest \emph{in} \MS\ \emph{and} \EDD
} aprapañcaṃ śīghram eva sthirībhavatīty ayaṃ viśeṣaḥ.\footnoteA{
	The sequence of the argumentation is off in \MS , with \emph{iti cet} and \emph{nanu} being added to two sentences that represent the \emph{siddhāntin}'s speech.
	Here the current \emph{siddhāntin} (probably equal to the author himself) is arguing against the criticism just expressed about the fourth \emph{sādhya}.

	\hspace*{1em}On can consider using the word \emph{atha} instead of \emph{nanu} in the sentence beginning \emph{nanu aprapañcaṃ śrīghram eva}, and one can also consider ending it with \emph{iti cet}, which may have accidentally been moved to the preceding sentence, and which may have a reflex in \TVB\ with \emph{zhe na}. 
	The flow of argumentation is somewhat less clear and certain in \TVA , which ends the sentence with \emph{'di khyad par yin te}.
	Although \emph{iti cet} is not strictly necessary here, what follows is certainly a response, attempting to show that a lack of \emph{prapañca} does not in fact lead to stability more quickly. This is clear from the conclusion: \emph{tasmān nāyaṃ viśeṣaḥ}.
}
yatraivālambane\footnoteB{
	yatraivālambane] \conj\ (\emph{no reflect of nanu} \emph{in} \TIB); nanu yatraivālambane \MS\ \EDD
} cittaṃ punaḥ punaḥ preryate nirantaraṃ\footnoteB{
	nirantaraṃ] \EDD\ (\emd) \TIB\ (rgyun mi 'chad par); niruttaraṃ \MS
} dīrghakālaṃ ca tatraiva sthirībhavatīty āgamo yuktiś cātrāsti.\footnoteB{
	°īty āgamaḥ. yuktiś cātrāsti] \MS\ \EDD\ (°ity āgamaḥ |) \EDD\ \TVB\ (zhes bya ba ni lung yin no || 'di la rigs pa yang yod de |); zhes bya ba ni lung yin no || 'di la rigs pa yang yod de | \TVB\ (°iti yuktiḥ. āgamaś cātrāsti)
} tathā coktam—

\begin{quote}
	tasmād bhūtam abhūtaṃ vā yad yad evābhibhāvyate | \\
	bhāvanābalaniṣpattau tat sphuṭākalpadhīphalam\footnoteB{
		kalpadhīphalam] \emd ; kalpadhīḥ phalam \MS\ \EDD
	}~||\footnoteA{
		\emph{Pramāṇavārttika}, Pratyakṣapramāṇa 285. The reading \emph{bhāvanābalaniṣpattau} is supported by the Tibetan translation and occurs in other sources (\emph{bsgom pa'i stobs ni rdzogs pa na}). The more mainstream reading for this \emph{pāda} is \emph{bhāvanāpariniṣpattau}.
	}
\end{quote}

punaś coktam—

\begin{quote}
	aho kusīdatvam aho vimūḍhatā\\
	aho janasyāsya sadarthavakratā |\\
	svacittamātrapratibaddhabuddhatā\footnoteB{
		°pratibaddha°] \conj\ (\TIB : 'brel pa); °pratibuddha° \MS\ \EDD
	}\\
	adūravartiny api yan na sevyate ||\footnoteB{
		Untraced. Also cited in *\emph{Saptāṅga} fol.\ 202r7. The verse is in Vaṃśastha metre.
	}
	% LGLGGLLGLGLG X 4
\end{quote}

\noindent iti. tasmān nāyaṃ viśeṣaḥ.

bhrāntirūpatvenāparamārthatvam api sarvatraiva bhāvanāviṣaye vastuni\footnoteB{
	bhāvanāviṣaye vastuni] \conj\ (\TVB : bsgom pa'i yul gyi dngos po); bhāvanāviśeṣe vastuni \MS\ \EDD ; bsgom pa'i yul gyis dngos po \TVA
} sambhavatīti na kiñcid api bhāvanīyaṃ syāt.\footnoteA{
	This sentence is significantly different in \TIB : \emph{spros pa la dmigs pa ni 'khrul pa'i ngo bo nyid kyis don dam pa ma yin pa nyid do zhe na | thams cad du bsgom pa'i yul gyis dngos po (mi) (sic \emph{for} ni?) 'khrul pas cung zhig kyang bsgom par bya ba med par 'gyur la |} \TVA ; \emph{'khrul pa'i ngo bo nyid kyis don dam pa ma yin pa nyid do zhes na | thams cad du bsgom pa'i yul gyi dngos po ni | 'khrul pa yin pas cung zhig kyang bsgom par bya ba med par 'gyur la |} \TVB .
	Apart from other minor differences, \TVA\ adds \emph{spros pa la dmigs pa ni} at the beginning of the sentence.
	The text could be rendered in Sanskrit as follows: \emph{(prapañcālambanasya) bhrāntirūpatvenāparamārthatvam iti cet, sarvatraiva bhāvanāviṣaye vastuni sambhrāntatvān na kiñcid api bhāvanīyaṃ syāt}.%
}
[\MS\ fol.\ 8v] tataś ca sarvatraiva mokṣamārge bhāvanāyā\footnoteB{
	mokṣamārge bhāvanāyā] \EDD ; mokṣamārge bhāvanāyāṃ \MS ; thar pa'i lam bsgom pa \TVA ; thar pa'i lam bsgom pa la \TVB\ (mokṣamārgabhāvanāyā)
} vaiyarthyaṃ syāt.
māyopamākārānupraveśena bhrāntirūpam apy\footnoteB{
	bhrāntirūpam apy] \MS\ \EDD\ \TVB\ ('khrul pa'i ngo bo la yang); de ltar 'khrul yang \TVA\ (evaṃ bhrāntam apy)
} aprapañcaṃ [\EDD\ p.\ 145] bhāvyamānam\footnoteB{
	aprapañcaṃ bhāvyamānam] \MS ; aprapaṃcā bhāvyamāṇam \MS ; aprapañcād bhāvyamānam \EDD ; spros pa med par bsgom par 'gyur ba \TVA ; spros pa med pa'i sgom par 'gyur ba \TVB
} aduṣṭaṃ bhavatīti cet, na tv ayaṃ māyākārānupraveśaḥ prapañce 'pi samāna iti.
tatrāpi ko doṣasyāvakāśaḥ?
tasmāt\footnoteB{
	tastmāt] \MS\ \EDD ; de bas na don du gnyer bas \TIB\ (tasmād arthī)
} prapañcam aprapañcaṃ vā yad eva rocate pramāṇasaṃgatam itarad vā, tad evālasyaṃ vihāya mahāmudrārthibhir\footnoteB{
	mahāmudrārthibhir] \conj\ (\TIB : phyag rgya chen po don du gnyer bas); mahāpuruṣārthibhir \MS\ \EDD
} bhāvayitavyam\footnoteB{
	bhāvayitavyam] \EDD ; bhaviyitavyam \MS
} ity alam atiprasaṅgeneti.

atra ca sāretaravibhāgaḥ paryupāsitagurubhir eva jñātavyaḥ.

% § 8.5_1
\subsubsection{pañcamasyāsāratvam}
\noindent tṛtīyapakṣoktadoṣatvān\footnoteB{
	tṛtīyapakṣoktadoṣatvān] \conj\ (\TVB : gsum pa'i phyogs la bshad pa'i nyes pa yod pa dang); tṛtīyapakṣe ktato \MS ; tṛtīyapakṣe kuto \EDD ; gsum pa'i phyogs la bshad pa'i nyes pa \TVA\ (\emph{see note} concerning \TVA )
} nīrasatvena\footnoteB{
	nīrasatvena] \conj ; nīrasatvena te \MS\ \EDD ; \emph{no reflext in} \TIB
} prayojanābhāvān mantranayakramābhāvāc ca\footnoteA{
	Both Tibetan translations lack a reflex of \emph{nīrasatvena}, but there is otherwise reason to assume the word to be an interpolation.
	While \TVB\ otherwise agrees with \MS , \TVA\ suggests a different structure to the text here: \emph{dgos pa la sogs pa gsum pa'i phyogs la bshad pa'i nyes pa dang | gsang sngags kyi tshul gyi rim pa med pa'i phyir}; \emph{prayojanābhāvāditṛtīyapakṣoktadoṣatvān mantranayakramābhāvāc ca}.
	It is true that \emph{prayojanābhāva} was an argument given against the third \emph{pakṣa}.
	Here, however, if that argument is further qualified with \emph{nīrasatvena}, its inclusion as a different reason is cogent—unlike the fifth \emph{pakñs}, the third includes bliss as an integral part of the \emph{sādhya}.
} na pañcamaḥ parikṣīṇadoṣaḥ.

% § 8.5_2
nanu sākṣātkaraṇāt\footnoteA{
	\TIB\ perhaps misinterprets \emph{sākṣātkaraṇāt pūrvaṃ} rather than reflects a differnet reading with \emph{sngar mngon du byas pa'i phyir} (\emph{sngon} is corrupted to \emph{sngon} in \TVD) (`before, for the sake of direct experience').
} pūrvaṃ mantranayaprayogo 'sti.
tat kathaṃ tasyābhāvaḥ?
satyam, sākṣātphalāvasthā sādhyā.
tasyāṃ ca nāsty asau kramaḥ sākṣāt. parityāge\footnoteB{
	kramaḥ sākṣāt. parityāge] \emph{variant word divion in} \MS\ \emph{and} \EDD : kramaḥ | sākṣātparityāge 
} ca na prayojanam utpaśyāma iti.\footnoteA{
	Both \TIB\ translations treat this section quite differently and may or may not suggest different underlying readings: \emph{bden te | bsgrub par bya ba 'bras bu mngon du gyur pa'i gnas skabs de yang rim pa 'di la med pa dang | mngon du gyur ba yongs su btang ba dang | dgos pa ma mthong ba'i phyir ro ||} \TVA ; \emph{bden te | bsgrub par bya ba 'bras bu mngon du gyur pa'i gnas skabs na de yang rim pa 'di la med pa dang | mngon sum du gyur pa yongs su btang ba la dgos pa yang ma mthong ba'i phyir ro} || \TVB .

	\hspace*{1em}The word \emph{sākṣāt} could be taken with \emph{parityāge} as \MS\ and \EDD suggest, but the point is not so much that the \emph{krama} is directly abandoned but that, according to this \emph{pakṣa}, it is not directly present when the direct result is directly experienced.
	This Vāgīśvarakīrti sees as rendering the \emph{sādhya} pointless.
}

% § 8.6_1
\subsubsection{ṣaṣṭhamasyāsāratvam}
\noindent svecchayā nirvāyayitum\footnoteB{
	nirvāyayitum] \MS ; nirvāpayitum \EDD
} aśakyatvāt, prayojanābhāvāt, sattvārthābhāvāc ca na pañcāntaraprabhedakalpanā\footnoteB{
	pañcāntara°] \emd\ \TIB\ (lnga pa'i mtha'i); prapañcāntara° \MS\ \EDD
} kalaṅkāśūnyā.\footnoteA{
	\TIB\ an alternative structure to the sentence: \emph{lnga pa'i mtha'i rab tu dbye ba rtog pa'i dri mas stong pa ma yin no ||} (no] \TVA ; te \TVB); \emph{na pañcāntaraprabhedaḥ kalpanākalaṅkāśūnyaḥ}.
}
tathā hi\footnoteB{
	tathā hi] \MS\ \EDD\ \TVB\ ('di ltar); de la ji ltar bzlog mi nus she na | \TVA\ (tatra kathaṃ nirvāyayituṃ na śakyata iti cet)
} kasyacin nivṛttiḥ kāraṇanivṛttyā vyāpakanivṛttyā\footnoteB{
	vyāpakanivṛttyā] \EDD ; vyāpakānivṛttyā \MS
} vā bhavati.
na cātra sākṣātkṛtamaṇḍalacakrasya nivartakaṃ kāraṇaṃ vyāpakaṃ vā icchākāle dṛśyate.\footnoteA{
	\TIB\ lacks a reflex of \emph{icchākāle dṛṣyate}.

	\hspace*{1em}Both translations add an extra sentence to this paragraph: \emph{rang gi 'dod pas ('dos pas \TVB ; 'gog par \TVB) 'gog pa yang mi nus te mi mthun pa med pa'i phyir | sdug bsngal la sogs pa 'gog pa 'dod kyang sdug bsngal la sogs pa la 'jug pa mthong ba'i phyir ro ||} `And it cannot be stoped by one's voliotion because [volition alone] is not doscordant with it. For, although they may desire to stop suffering and the like, it is observed that people continue to engage in suffering'.

	\hspace*{1em}Given the Tibetan text, it is possible there is an omission between \emph{vyāpakaṃ vā} and \emph{icchākāle}.
	The words \emph{icchākāle dṛśyate} are strictly speaking not necessary in this sentence, but they are also not inappropriate: according to this position, it is \emph{icchā} that is the occasion for the cessation of the \emph{maṇḍala}.
}

% § 8.6_2
nanu śūnyataiva nivartikāsti.
yathā dārusaṅghātaprajvalito\footnoteB{
	dārusaṅghātaprajvalito] \conj ; dārusaṃghāt pravjalito \MS ; dārusaṃghāte prajvalito \EDD
}\footnoteA{
	Where \MS\ reads \emph{dārusaṃghāt pravjalito} we conjecture the compound reading \emph{dārusaṅghātaprajvalito} on the strength of \emph{maṇḍalacakraprajvalitaḥ} below.
	\TIB\ renders both compounds somewhat freely and does not clearly help in deciding the matter.
} vahnir niḥśeṣam indhanaṃ bhasmīkṛtya paścāt svarasata eva nivartate, tathā maṇḍalacakraprajvalitaḥ śūnyatājñānāgniḥ sākṣāt kṛtvā\footnoteB{
	sākṣāt kṛtvā] \conj ; sākṣān \MS\ \EDD ; mngon sum du byas nas kyang \TIB\ (sākṣāt kṛtvāpi)
} maṇḍalacakraṃ nivartayiṣyatīti cet.\footnoteA{
	\TIB\ a fuller sentence here. \TVB\ reads: \emph{de ltar dkyil 'khor gyi 'khor lo stong pa nyid kyi ye shes kyi me rab tu 'bar bas mngon sum du byas nas kyang | dkyil 'khor gyi 'khor lo ma lus par ldog par byed la | bdag nyid kyang rang gi ngang gis ldog par 'gyur ro zhe na |}; \ldots\ \emph{kṛtvāpy aśeṣamaṇḍalacakraṃ nivartyitvā svam api svarasato nivartate}.

	\hspace*{1em}\TVA\ appears to be slightly more corrupt, but suggests roughtly same text: \emph{de dkyil 'khor gyi 'khor lo stong pa nyid kyi ye shes kyi me rab tu 'bar bas mngon sum du byas nas kyang | dkyil 'khor gyi 'khor lo ma lus par ldog par byed la | de yang rang gi ldog par 'gyur ro zhe na |} 
} tad asat, viṣamatvād dṛṣṭāntasya.
tathā hi tatrendhanaṃ kāraṇaṃ\footnoteB{
	kāraṇaṃ] \conj ; na kāraṇaṃ \MS\ \EDD
} vahneḥ.
kāraṇasyendhanalakṣaṇasya nivṛttau\footnoteB{
	kāryasyendhanalakṣaṇasya nivṛttau] \conj ; kāryam indhanalakṣaṇanivṛttau \MS\ \EDD
} yuktaiva vahnilakṣaṇasya kāryasya nivṛttiḥ.
iha tu na śūnyatā kāraṇaṃ maṇḍalacakrasya.
tat ka[\MS\ fol.\ 9r]thaṃ tannivṛttau nivṛttiḥ?
na\footnoteB{
	na] \conj ; athavā na \MS\ \EDD
} ca śūnyatāyā nivṛttir asti.\footnoteA{
	\TIB\ diverges significantly in this section: \emph{'di ltar de la shing 'gyur bar byed pa'i rgyur gyur pa'i me ni shing 'gyur ba yang dag par skyed pa nyid kyis | shing 'gyur bar byed pa ni me yin no zhes 'jig rten pa rnams sems la | 'dir ni stong pa nyid ni dkyil 'khor gyi 'khor lo 'gyur bar byed pa'i rgyu ma yin na | de'i phyir de ldog par byed pa yin | stong pa nyid la rang gi ngang gis ldog go zhes kyang smra bar bya ba ma yin no ||}.

	\hspace*{1em}\TVA\ is again mostly the same text with a few more minor corruptions: \emph{de la 'di ltar shing 'gyur bar byed pa'i rgyur gyur pa'i me ni shing gi 'gyur ba yang dag par skyed pa nyid kyis shing 'gyur bar byed pa ni me yin no zhes 'jig rten pa rnams sems la 'dir na stong pa nyid kyi dkyil 'khor los sgyur bar byed pa'i rgyu ma yin na de'i phyir ldog par byed pa yin no || de ci'i phyir zhe na | stong pa nyid la rang gi ngang gis ldog go zhes kyang smra bar bya ba ma yin no ||}.

	\hspace*{1em}Although \MS\ is also quite corrupt in this paragraph, it is difficult to see how the text it transmits corresponds to the Tibetan translation.
	It is also not obvious what Sanskrit potentially lay behind this Tibetan translation.
	\TVB\ can be rendered into English as follows: `To explains, in [the example], fire, as the cause of transmormation in wood, creates a transformation in wood, and by this common people believe that fire transforms wood. In this case, however, emptiness is not a cause for a transformation in the \emph{maṇḍala}. Therefore, one cannot say that this [emptiness] is a cause of cessation, nor that emptiness will cease of its own accord.'
}

% § 8.6_3
nanu sā na\footnoteB{
	na] \EDD\ (\emd); deest \emph{in} \MS
} bhavatu kāraṇaṃ.
śūnyatā vyāpakaṃ tu bhaviṣyati.
vyāpakasya vṛkṣasya nivṛttau śiṃśapātvasya vyāpyasya nivṛttivan nivṛttir bhaviṣyatīti cet.
etad apy asāram.
tathā hi śūnyatā sarvadā\footnoteB{
	sarvadā] \MS\ \EDD\ \TVA\ (sarvadā); \emph{no reflex in} \TVB
} sarvajñeyamaṇḍalavyāpikā tattvarūpā.\footnoteB{
	tattvarūpā] \EDD ; tatvarūpāḥ \MS
} na ca tasyā\footnoteB{
	tasyā] \MS\ \EDD ; \emph{no reflex in} \TIB
} nivṛttiḥ kadācid apy asti.
yadi syāt\footnoteB{
	syāt] \MS\ \EDD\ \TVB ; ldog par 'gyur na \TVA\ (nivṛttiḥ syāt)
} samyaksaṃbodhisākṣātkaraṇāt [\EDD\ p.\ 146] pūrvam anantaram eva vā nivṛttiḥ\footnoteB{
	nivṛttiḥ] \MS\ \EDD ; sangs rgyas bcom ldan 'das ldog par 'gyur na \TIB\ (bhagavataḥ buddhasya nivṛttiḥ)
} syāt.
na ca bhavati, samyaksaṃbuddhībhūyāpi katipayakālāvasthānasya svayam eva svīkṛtatvāt.\footnoteB{
	svīkṛtatvāt] \MS\ \EDD ; zhal gyis bzhes pa'i phyir 'gyur ba yang ma yin no \TIB\ (na ca bhaviṣyati [?])
}

% § 8.6_4
kintu śūnyatāpi jñānarūpā, cakram api jñānarūpam.
śūnyatājñānotpattyā cakrajñānasyānivṛttau\footnoteB{
	°ānivṛttau] \MS\ \EDD ; log na \TIB\ (nivṛttau)
} śūnyatājñānaṃ kena nivartanīyam?
tena nivṛttiś ca virodhino 'bhāvāt kāraṇavyāpakayoś cābhāvān nāsti.
tasmāc chūnyatājñānasya na nivṛttiḥ,\footnoteB{
	na nivṛttiḥ] \conj\ (\TIB : ldog pa med do); nivṛttiḥ \MS\ \EDD
} nāpi maṇḍalacakrasya śūnyatāto nivṛttir iti śūnyatā na nivartikā.\footnoteA{
	\TVB\ reflects closely the Sanskrit text transmitted in \MS , but \TVA\ diverges significantly start from \emph{tena nivṛttiś}: \emph{de nyid kyis ldog pa ni 'gal ba'i phyir la | rgyu'am khyad par byed pa med pa de bas na stong pa nyid kyi ye shes la ldog pa med do || de bas na dkyil 'khor gyi 'khor lo yang stong pa nyid kyis ldog go zhes bya ba yang ma yin te |}.
	Given that the syntax and logic of the passage is far from clear, the translation has probably suffered corruption in a number of places (e.g., \emph{khyad par} for \emph{khyab pa} and possibly \emph{'gal ba'i phyir} for \emph{'gal ba med pa'i phyir}). 
}

% § 8.6_5
ko brūte śūnyatā nivartikā?\footnoteA{
	Here again \TVA\ divergers from \MS\ significantly, and transmits a text that is not easily comprehensible: \emph{stong pa nyid la ldog par byed pa yin pa | log pa dang stong pa nyid kyis ldog par byed pa yin no zhes su zhig smra |}
}
kiṃ tarhi yan nivartakaṃ\footnoteB{
	nivartakaṃ] \emd ; nivartikās \MS\ \EDD
} tad gurūpadeśato jñeyam ity apy asāram.
gurūpadeśato 'pi na śūnyatāvyatiriktaṃ\footnoteB{
	°vyatiriktaṃ] \conj ; vyatiri((ktiḥ)) \MS\ (i \emph{in} kti \emph{lacks a} pṛṣṭhamātrā); °vyatiriktaḥ \EDD
} pramāṇato 'stīti yatkiñcid etat.\footnoteA{
	\TIB\ does not reflect the Sanskrit of \MS\ for this sentence: \emph{bla ma'i man ngag las kyang stong pa nyid dang | de ldog pa las ma gtogs pa'i ldog par byed pa'i tshad ma gzhan cung zad yod pa ma yin no ||} (\TVA);
	\emph{bla ma'i man ngag las kyang stong pa nyid kyis ldog par byed pa ma yin ldog pa'i tshad ma cung zhig kyang yod pa ma yin pas} (\TVB). 
	Again \TVA\ appears corrupt and barely cohrenet.
	\TVB\ is understandable and somewhat related to the overall meaning of the Sanskrit: `From a guru's instruction there is not even the slightest means of knowledge for a cessation that is not a cessation by means of emptiness; therefore, \ldots '
} pratikṣaṇanivṛttiś ca kṣaṇabhaṅgarūpā sarvapadārthavyāpinī.
na sā santānanivartikā.
tasmān na svecchayā nivṛttiḥ.\footnoteB{
	nivṛttiḥ] \MS\ACreading\ \EDD ; nivṛrttiḥ \MS\PCreading ; brtag par [mi bya ste] \TVA\ (sic \emph{for} ldog par [mi bya]?); \emph{no reflex in} \TVB 
} na ca nivṛttyā\footnoteB{
	nivṛttyā] \EDD\ (\emd); nivartyā \MS
} nīrasarūpayā prayojanam asti prekṣāvatām.\footnoteA{
	This sentence too appears differently in \TVA : \emph{ldog pa'i snying po med pa la rtog pa dang ldan pa rnams kyis brtags pa dgos pa yod pa ma yin no ||}; `There is no need for reflective people to examine insipid cessation.'
} tathā coktam\emdash 

\begin{quote}
	mucyamāneṣu sattveṣu ye te\footnoteA{
		The pronound combination \emph{ye te} can have the sense of "whatever there may be" (\cite[§287]{speijer1886}). In this case, Prajñākaramati in his \emph{Bodhicaryāvatārapañjikā} interprets them as conveying the sense of inexpressibility: \emph{ye te iti teṣām eva anubhavasiddhatvād idaṃtayā kathayitum aśakyāḥ} (p.\ 341).
		% de bas na bstan pa'i phyir/ sems can rnams ni grol ba na/ /zhes gsungs te/ sems can rnams la sdug bsngal gyi 'ching ba las grol bar byed pa yod pa la'o/ /gang yin de zhes pa ni de rnams nyid kyis nyams su myong bas grub pa'i phyir 'di tsam zhes brjod par nus pa ma yin la/ de nyid kyi phyir dga' ba'i rgya mtsho yin te mgu ba'i rgya mtsho snying rje dang ldan pa'i rgyud rnams la skye bar 'gyur ro/ /dga' ba'i rgya mtsho de nyid kyis mthar phyin pa ni bde ba gzhan la phyir phyogs pa las yongs su thob pa ste/ bde ba thams cad spangs pa yin no zhes bya ba'i tha tshig go/ /ci ste de nyid bde ba'i mthar thug pa ci ltar yin te/ gcig tu nyon mongs pa'i yongs su gdung ba nye bar zhi bas rnam par grol ba'i bde ba yin pa'i phyir shin tu lhag pa yin no zhe na/ ro bral thar pa zhes pa la sogs pa gsungs te/ thar pa yang de la ltos nas de rnams ro dang bral ba nyid du snang ngo/ /dgongs pa ni 'di yin te/ gzhan la phan pa dang bde ba bsgrub pa nyid de rnams kyi ro'i mchog nyid kyis zhugs pa'i phyir de la mngon par zhen pa yin la/ snying rje dang ldan pa rnams la bde ba de nyid dgos pa yin gyi thar pa ni ma yin no/ /
		%
		% Steinkellner: Have not those who are oceans of joy when beings are liberated accomplished their task through these alone? What would be the use of a liberation not relished? 
		%
		% Understand only two sentences, takes prāmodyasāgarāḥ as a metaphorical compound.
		%
		% duḥkhabandhanādvisaṃyujyamāneṣu sattveṣu satsu| ye te iti| teṣāmeva anubhavasiddhatvādidaṃtayā kathayitumaśakyāḥ, ata eva prāmodyasāgarāḥ saṃtuṣṭisamudrāḥ kṛpāvatāṃ saṃtāneṣu prādurbhavanti| taireva prāmodyasāgaraiḥ paryāptaṃ tadanyasukhavaimukhyāt parisamāptam| *******
		% 
		% Canonical BCA translation: sems can rnam par grol ba na/ /dga' ba'i rgya mtsho gang yin pa/ /de nyid kyis ni chog min nam/ /thar pa 'dod pas ci zhig bya/ /
 	} prāmodyasāgarāḥ | \\
	tair eva nanu paryāptaṃ mokṣeṇārasikena kim ||\footnoteA{
		\emph{Bodhicaryāvatāra} 8.108
	}
\end{quote}

\noindent iti.

% § 8.6_6
sattvārtho 'pi nivṛttau nāsti.
na hi gagane\footnoteB{
	gagane] \MS\ \EDD\ \TVB\ (nam ka); \emph{no reflext in} \TVA
} gaganakamale vā kācid arthakriyā sambhavati.
ciraniruddhād apy atītād avasturūpāc\footnoteB{
	avasturūpāc] \MS\ \EDD\ \TVB\ (dngos po med pa'i ngo bo); dngos po'i ngo bo \TVA\ (vasturūpāc)
} cakrāt sattvārtho bhaviṣyatīty apy asāram, ciranīrutasyāpi kukku[\MS\ fol.\ 9v]ṭasya\footnoteB{
	ciranīrutasyāpi kukkuṭasya] \conj ; cirutasyāpi kukkuṭasya \MS ; ciravirutasyāpi kukkuṭasya \EDD ; yun rin por khyim bya shi ba \TVA ; yun ring por lon pa'i khyim bya shi ba \TVB\ (ciramṛtasyāpi kukkuṭasya)
}\footnoteA{
	The conjecture \emph{ciranīrutasyāpi} (`long-silent' or `long-since mute') is paleographically closer to \MS 's \emph{cirutasyāpi}, though \TIB 's suggested \emph{ciramṛtasya} (`long-dead') offers a less ambiguous example.
} kaṇṭhadhvaniprasaṅgāt.

% § 8.6_7
nanu yogyadhiṣṭhānād gaganād apy arthakriyāḥ sambhavantīti cet.\footnoteB{
	sambhavantīti cet] \conj ; saṃbhavanti \MS\ \EDD
} na sambhavanti, yogyadhiṣṭhānād eva cittarūpād arthakriyā, na gaganāt, nīrūpatvāt tasya.\footnoteA{
	\TVA\ varies significantly for this paragraph and is again not readily understandable: \emph{rnal 'byor pa'i byin gyi rlabs kyis nam mkha' las kyang dngos po'i ngo bo 'das pa'i 'khor lo las sems can gyi don byed pa yin la | nam mkha' ni ma yin te | de'i ngo bo nyid ma yin pa'i phyir ro ||}
}

% § 8.6_8
nanu nirodhya maṇḍalacakraṃ sattvārthakāle punar utpādyate.
tato 'rthakriyā bhavati.
tataḥ punar eva nirodhyate, punar evotpadyata iti cet.
asad etat.
yathā sattvārthakriyāyās tattvato\footnoteB{
	tattvato] \MS\ (tatvato) \EDD ; de las \TIB\ (tato)
} nāsti prādurbhāvaḥ, tathā cakrasyāpi.
tato nārthakriyāyāḥ sambhavaḥ.\footnoteA{
	From \emph{yathā sattvā°} to \emph{sambhavaḥ}, \TIB\ is significantly different: \emph{ji ltar sems can gyi don byed pa de las byung ba med pa de bzhin du don byed pa'i 'khor lo yang de la mi srid do ||} (\TVA); \emph{ji ltar sems can gyi don byed pa de las 'byung ba med pa de bzhin du 'khor lo yang de las mi srid do ||} (\TVB)

	\hspace*{1em}It is likely that both translations reflect \emph{tato} in place of \emph{tattvato}, but it is is unclear what this pronoun would refer to.
	The reading \emph{tattvatas} can be understood in the sense that accomplishing the aims of beings does not manifes spontaneously from the nature of reality, nor does the \emph{cakra}; hence, this position cannot explain how the \emph{cakra} and hence \emph{sattvārtha} are made to restart after cessation.
}
na ca nirodhya\footnoteB{
	nirodhya] \EDD ; niro((dhya)) \MS\ (\emph{some kind of correction is made, but uncertain from what to what}); 'gogas pa las (\emph{possibly} nirodhāt)
} punar utpāde kiñcit prayojanam astīty alam atiprapañceneti.

% § 8.7
\subsubsection{saptamasyāsāratvam}
\noindent ṣaṣṭhapakṣoktadoṣasandohasya saptame\footnoteB{
	ṣaṣṭhapakṣoktadoṣasandohasya saptame] \conj\ (\TVB : drug pa'i phyogs la bshad pa'i skyon gyi tshogs bdun pa la) (\TVA : \emph{samd as} \TVB , \emph{with} gyis \emph{for} gyi); ṣaṣṭhapakṣoktaṃ saṃdāhasyāṣṭame \MS ; ṣaṣṭhapakṣoktasaṃdohasyāṣṭame \EDD
} 'pi bhāvān na piṣṭapeṣaṇaṃ\footnoteB{
	piṣṭapeṣaṇaṃ] \MS\ACreading\ \EDD ; piṣṭapre | ṣaṇaṃ \MS\ACreading
} kriyate.
nanu ṣaṣṭhena saptamasya samānatvāt kathaṃ saptamasya tato viśeṣaḥ?\footnoteB{
	dsaptamasya tato viśeṣaḥ] \MS\ \EDD\ \TVB\ (de las bdun pa khyad par ci yod); de la khyad par ci yod \TVA\ (tatra ko viśeṣaḥ)
}
asti viśeṣaḥ.
pūrvāvasthāyāṃ niyatacakrākāratā, punaḥ svecchayā nirodhaḥ svecchayotpādanaṃ\footnoteB{
	nirodhaḥ svecchayotpādanaṃ] \conj\ (\TVB : yang rang gi 'dod pas 'gog cing rang gi 'dod pas skyed par byed pa); svecchetpādanaṃ \MS ; svecchotpādanaṃ \EDD ; rang gi 'dod pas skyed par byed pa nyid \TVA\ (svecchayotpādanaṃ)
} ca.\footnoteB{
	ca] \conj ; ceti \MS\ \EDD
}\footnoteA{
	The reading of \MS , \emph{punaḥ svecchayā svecchetpādanaṃ}, suggests that a word dropped after \emph{svecchayā}, and \TVB\ supplies a fitting word (\emph{'gog} / \emph{nirodha}).
	Surprisingly \TVA\ lacks a reflex of \emph{svecchayā nirodhaḥ}, but without this the text does not sound complete: \emph{yang dang yang du rang gi 'dod pas skyed par byed pa nyid yin la}; \emph{punaḥ puunaḥ svechayotpādanaṃ}.
	This perhaps represents a reading that was an early corruption in the textual transmission.
}
saptame punar etan nāsti.
tato na samānatā.
bhinnaś ca nirdiṣṭa iti.\footnoteA{
	\TIB\ reflects an alternative but mistaken interpretation of the text's word division: \emph{de bas na mtshungs pa dang tha mi dad pa ma yin par bstan to ||}; \emph{tato na samānatābhinnaś ca nirdiṣṭa iti}.
}

% § 9
\subsection{caturthasya sekasya svarūpam}
\begin{quote}
	dambholibījasrutidhautaśuddha-\footnoteB{
		°sruti°] \corr ; śruti \MS\ \EDD
	}\\
	pāthojabhūtāṅkurabhūtapuṣṭi\footnoteB{
		pāthoja°] \EDD\ (\emph{\EDD reports the ms.\ as reading \emph{pāthojña}, but this seems to be incorrect}); pāthauja° \MS
	}|\\
	turīyaśasyaṃ\footnoteB{
		turīyasasyaṃ] \EDD\ (turīyaśasyaṃ); tutīyaśasyaṃ \MS
	} paripākam eti\footnoteB{
		eti] \EDD\ (\emd); eta \MS
	} \\
	sphuṭaṃ caturthaṃ viduṣo 'pi gūḍham || 17 ||
\end{quote}

\noindent [\EDD\ p.\ 147] dambholītyādi.
etat sadgurūpadeśato jñeyam.

\subsection{aparāṇi mithyāsādhyāni mithyātattvāni ca}
\begin{quote}
	pañcapradīpāmṛtabinducandra-\\
	bhrūmadhyabindūdbhavamaṇḍalāni |\\
	vāyoḥ svarūpaṃ galaśuṇḍikādyam \\
	atattvarūpaṃ svayam ūhanīyam || 18 ||\footnoteA{
		This verse is in Upajāti metre.
	}
\end{quote}

% § 10_1
\noindent pañcapradīpetyādi.
pañcapradīpaśabdena gokudahanalakṣaṇasya, amṛtaśabdena\footnoteB{
	amṛtaśabdena] \MS\ \EDD ; bdud rtsi lnga'i sgra ni \TIB\ (pañcāmṛtaśabdena)
} vimumāraśulakṣaṇasya satatānuṣṭhānam eva sādhyaṃ manyante.
bindur iti hṛccandrasthaṃ binduṃ dedīpyamānaṃ tattvaṃ sādhyaṃ ceti kṛtvā kecid bhāvayanti.
candra iti hṛdisthaṃ kalārūpam ardhacandraṃ vā\footnoteB{
	kalārūpam ardhacandraṃ vā] \MS\ \EDD ; zla ba phyed pa'am | zla ba rgyas pas \TVA\ (ardhacandraṃ pūrṇacandraṃ vā); zla ba'i bzhi dum bu'am zla ba phyed pa'am | zla ba rgyas pa \TVB\ (kalārūpam ardhacandraṃ pūrṇacandraṃ vā)
} hṛtkamalasthaṃ kecid bhāvayanti.

% § 10_2
bhrūmadhyabindūdbhavamaṇḍalānīti bhruvor madhya ūrṇāyāṃ binduṃ vibhāvya tadbindūdbhavāni maṇḍalāni vāyuvāruṇamāhendrāgneyalakṣaṇāni.
etad uktaṃ bhavati\emdash mukhaśravaṇacakṣurghrāṇarasanāni\footnoteB{
	mukhanāsikācakṣurghrāṇarasanāni] \conj ; mukhaśravaṇanāsikācakṣurghrāṇarasanāni \MS\ \EDD ; kha dang | rna ba dang | sna dang | mig \TVA\ \TVB
}\footnoteA{
	It what elements should be included in this compound, given \TIB\ omits the tongue and \MS\ includes the nose twice.
	Provisionally we adopt a reading with only one instance of the nose.
} hastāṅgulībhiḥ pidhāya bhrūmadhyabindur draṣṭavyaḥ.
tasya sphuṭāvasthāyāṃ śubhāśubhani[\MS\ fol.\ 10r]\-mittasaṃsūcakāni māhendrādimaṇḍalāny upajāyante.
taṃ ca binduṃ tattvam iti manyante.

% § 10_3
vāyoḥ svarūpam iti pūrakakumbhakarecakapraśāntakalakṣaṇam\footnoteB{
	°recaka°] \EDD ; recakaṃ \MS
} ānāpānādilakṣaṇaṃ\footnoteB{
	ānāpānādi°] \EDD ; anāpānādi° \MS
} ceti.
etad\footnoteB{
	etad] \EDD\ (\emd); tad \MS
} uktaṃ bhavati\emdash śaivasāṃkhyādinirdiṣṭaṃ\footnoteB{
	śaivasāṃkhyādi°] \EDD\ (\emd) \TVB\ (shi ba dang grangs can la sogs pas); saivasaṃkhyādi° \MS ; grangs can la sogs pas \TVA\ (sāṃkhyādi°)%
} vāyusvarūpaṃ jñātvā taṃ vāyuṃ nirodhabhāvanayā\footnoteB{
	nirodhabhāvanayā] \MS\ \EDD\ \TBV\ ('gag pa'i sgom pa); bsgags pa las \TVA\ (nirodhena)%
} sthirīkṛtyākāśa utplutya\footnoteB{
	ākāśa utplutya] \conj ; ākāśenotplutya \MS\ \EDD
} gamanaṃ parapurapraveśaṃ yāvan muktiṃ ca sākṣāt kurvanti vāyuvādinaḥ. 

% § 10_4
galaśuṇḍiketi.
galapradeśe jihvāmūlopari hastiśuṇḍikākārā\footnoteB{
	hastiśuṇḍikākārā] \MS\ \EDD ; hastiśuṇḍākārā \possibleemd
} adhaḥpralambamānā upajihvāsaṃjñikā galaśuṇḍikāsti.
sā ca śaktirūpā.
tadadhaḥ śivarūpam\footnoteB{
	tadadhaḥ śivarūpam] \MS\ \EDD\ \TVB\ (de'i 'og na zhi ba'i ngo bo); sdig pa'i rang bzhin du yong pa \TVA
} asti tattvam.
sā ca [\EDD\ p.\ 148] jihvāgreṇa spṛśyamānā nirantarāmṛtaṃ sravati.\footnoteA{
	Like the English verb 'to flow', the Sanskrit √\emph{sru} is ambitransitive, although appears to be intransitive in greater frequency.
}
tena ca ghargharāmṛtavarṣaṇena santarpyamānam ātmānaṃ\footnoteB{
	ātmānaṃ] \MS\ \EDD ; don dam pa'i bdag nyid \TIB\ (pāramārthikam ātmānaṃ)
} dhyāyād iti galaśuṇḍikātattvam.
ādiśabdena hṛnmadhyaṣoḍaśanāḍikācakramadhyasthajñānasvarūpaṃ\footnoteB{
	hṛnmadhyaṣoḍaśanāḍikācakramadhyasthajñānasvarūpaṃ] \MS\ \EDD\ \TVB\ (snying ka'i dbus kyi 'khor lo rtsibs bcu drug pa'i dbus na gnas pa ye shes kyi rang bzhin); snying ga'i dbus kyi dkyil 'khor rtsibs bcu drug pa'i dbus na hūṃ gnas pa ye shes kyi rang bzhin (hṛnmadhyaṣoḍaśanāḍikāmaṇḍalamadhyahūṁsthajñānasvarūpaṃ)
} śivarūpaṃ tattvaṃ bhāvayitavyam ityādīnāṃ parigrahaḥ.\footnoteA{
	\TIB\ continues to describe this practice. \emph{yang smras pa | bcu las drug lhag rtsa dang ldan pa'i 'khor lo yi || dkyil na gnas pa'i snying gar rnam par gnas pa'i bdag | des ni de yi khyad par lta bu'i grub pa ster || de ni mngon par mi g-yo ba yi yid dag gis || rnal 'byor pa yi sems de de ltar mngon par bsam || nub par gyur pa'i mgon po rgyal bar gyur de ni || nus pa dag gis de ni yongs su bskor dang bcas ||} (\TVA). \emph{de yang smras pa | bcu las drug lhag rtsa dang ldan pa'i 'khor lo'i dkyil na gnas pa snying kar rnam par gnas pa'i bdag | des ni de'i khyad par lta bu yi grub pa ster | de ni mngon par mi g.yo ba'i yid dag gis || rnal 'byor pa yis de ltar mngon par bsam par bya || nus par gyur pa'i mgon po rgyal bar gyur || de ni nus pa dag gis de ni yongs su bskyor dang bcas ||} (\TVB)
}

tat sarvaṃ tīrthikādibhis tattvarūpeṇābhimatam atattvam iti svayam evohanīyaṃ vicāraṇīyam iti yāvat.

\subsection{upasaṃhāraḥ}
\begin{quote}
	svapnendrajālapratibimbamāyā-\footnoteA{
		Although not entirely certain, the commentary, by separating the first three words into a compound, may suggest reading \emph{svapnendrajālaiḥ} instead of \emph{svapnendrajāla°}.
	}\\
	marīcigandharvapurāmbu{[}\MS\ fol.\ 2r{]}candraiḥ |\\
	anyaiś ca śabdair\footnoteB{
		śabdair] \emd\ (TaRaA-Vi); sarvair \MS\ \EDD
	} upamābhidheyair\footnoteA{
		\TIB\ lacks a reflex of \emph{upamā} in \emph{upamābhidheyair} (\emph{mngon par brjod pa yis}) as well as in the commentary when translating \emph{upamābhidheyair upamāvācakair} (\emph{mngon par brjod pa ni smra ba pos}).
	} \\
	naivāsti sādhyaṃ kathitād ihānyat || 19 ||\footnoteA{
		This verse is in Upajāti metre.
	}
	% Vāṇi: GGLGGLLGLGG, LGLGGLLGLGG, GGLGGLLGLGG, GGLGGLLGLGG
\end{quote}

% § 11_1
\noindent svapnendrajāletyādi.
svapnendrajālopamaṃ pratibimbamāyāmarīcigandharvanagarodakacandropamam iti śabdaiḥ, anyaiś ca gaganapratiśrutkaphenopamam\footnoteB{
	gagana°] \corr ; gagaṇa° \MS\ \EDD
} ityādiśabdair upamābhidheyair upamāvācakair naivāsti sādhyaṃ kathitāt sādhyād anyat.
paraṃ iha kathita\footnoteB{
	iha kathita] \conj\ (\TIB : 'dir bshad pa); kathita \MS\ \EDD 
}\footnoteA{
	Strictly speaking the text is not incorrect as \MS\ reads without \emph{iha}, but its absence here is stricking given that it is found in the root text and that it is important in precisely conveying the intended meaning.
} eva sādhya ete śabdāḥ pravartanta iti svayaṃ boddhavyam.

\begin{quote}
	gambhīraśūnyapratibhāsamātra-\footnoteB{
		°mātra°] \EDD ; mātraṃ \MS
	}\\
	śāntāti\footnoteB{
		śāntāti°] \EDD ; sāntādi° \MS
	}sūkṣmānabhilāpyaśabdaiḥ |\\
	nirlepanīrūpa\footnoteB{
		nirlepanīrūpa°] \EDD\ (\emd); nirlepanīpa \MS
	}nirañjanādyair \\
	bhrāntir na kāryāparasādhyasattve || 20 ||\footnoteB{
		This verse is in Indravajrā
	}
\end{quote}

% § 11_2
\noindent [\EDD\ p.\ 149] gambhīraśūnyaṃ pratibhāsamātraṃ śāntātisūkṣmam anabhilāpyaṃ nirlepaṃ nīrūpam\footnoteB{
	nīrūpam] \EDD\ (\emd); nirupamaṃ \MS
} nirañjanādi.\footnoteB{
	nirañjanādi] \MS ; nirañjanaṃ \EDD
}\footnoteA{
	From \emph{ghabhīra°} to \emph{nirañjanādi} \TIB\ reproduces the root text.
} ādiśabdāt śivaṃ nirākāraṃ niṣprapañcam anādyantanidhanam i[\MS\ fol.\ 10v]\hspace*{0em}tyādiśabdair bhrāntir na kartavyāparasādhyasattva aparasya sādhyasya sattve sattāyām.\footnoteB{
	sattāyām] \MS ; sattvāyā \EDD
} ebhiḥ sarvair\footnoteB{
	sarvair] \MS\ \EDD ; sgra \TIB\ (śabdair)
} eva param api kiñcit sādhyaṃ kathitād astīti bhrāntir na kartavyā.
atha nātikathitam eva sādhyam ebhiḥ sarvair abhidhīyata iti niścayaḥ.

% § 12
\subsection{pariṇāmanā}
\begin{quote}
	akhilagaganagarbha\footnoteB{
		°gaganagarbha°] \corr ; °gagaṇagarbha° \MS\ \EDD
	}vyāpi\footnoteA{
		One can either read \emph{°gaganagarbhavyāpi} in compound with \emph{saptaprakāra°} or as qualifying \emph{puṇya}.
		It seems likely that the author intended this ambiguity.
	} saptaprakāra-\footnoteB{
		°saptaprakāra°] \EDD ; °sarvaprakāra° \MS
	}\\
	grathitavacanarūpād yan mayāsādi puṇyam |\\
	anupamasukhavidyāsaktasaddehanirmij-\footnoteA{
		It would appear that \emph{nirmij} (or \emph{nirmit}) is \emph{metri causa} for \emph{nirmimat}. 
	}\\
	jinajanitajanārthas tena loko 'yam astu || 21 ||\footnoteA{
		This verse is in Mālinī metre.
	}
	% ā√-sad in the sense of to obtain, passive aorist = āsādi

	tattvaratnāvalokaḥ samāptaḥ. kṛtir iyaṃ paṇḍitavāgīśvarakīrtipādānām.\\
\end{quote}

\setlength\parindent{0pt}
śrīsamāje parā yasya bhaktir niṣṭhā ca\footnoteA{
	Given the position of \emph{ca}, it seems that we should take \emph{niṣṭhā} as a substantive rather than an adjective qualifying \emph{bhakti}. \TIB , somewhat unnaturally, reflects understanding \emph{niṣṭhā} as an adjective meaning `perfected' by rendering it \emph{mthar phyin pa}.
} nirmalā |\\
tasya vāgīśvarasyeyaṃ kṛtir vimatināśinī\footnoteB{
	vimatināśinī] \EDD ; vimatināsanī \MS
} ||\\

vikacakumudatārākṣīrakundānukāri\footnoteB{
	vikacakumudatārākṣīrakundānukāri] \emd ; vikacakumudakṣīratārakundānukāri \EDD ; vikarektāmudakṣīratārākundānukāri \MS
}\footnoteA{
	The slight rearrangment of word order, from \emph{°kṣīratārā°} to \emph{°tārākṣīra°}, corrects the metre of the verse, which is Mālinī.
	Note, however, that \TIB\ perhaps reflects \MS 's word order: \emph{'o ma skar ma lta bu} (\TVA); \emph{'o ma lta bur skar ma lta bu'i} (\TVB).
}\\
pracitam api ca puṇyaṃ yan mayā granthito 'smāt |\\
anupamasukhapūrṇaḥ svābhavidyopagūḍho\\
bhavatu nikhilalokas tena vāgīśvaraśrīḥ ||\\

tattvaratnāvalokavivaraṇaṃ samāptam.
kṛtir iyaṃ paṇḍitācāryavāgīśvarakīrtipādānām.\\

\section{References}

\EmbracOff
\emph{Abhidharmakośa} by Vasubandhu. \fullcite*{pradhan1975}\mybibexclude{pradhan1975}\\

\emph{Abhidharmakośavyākhyā} by Yośamitra. \fullcite*{wogihara1932to6}\mybibexclude{wogihara1932to6}\\

\emph{Amṛtakaṇikā} by Raviśrījñāna. \fullcite*{lal1994}\\

\emph{Nyāyabindu} by Dharmakīrti. \fullcite*{scerbatskoj1918}\mybibexclude{scerbatskoj1918}\\

\emph{Pramāṇavārttikavṛtti} by Manorathānandin. Sāṃkṛtyāyana ed.\\

\emph{Mantrārthāvalokinī} by Vilāsavajra. \fullcite*{tribe2016}\mybibexclude{tribe2016}\\

\emph{Mitākṣarā} of Vijñāneśvara. \fullcite*{acharya1949}\mybibexclude{acharya1949}\\

\emph{Vṛttamālāvivṛti} by Śākyarakṣita. Hahn ed.\\

\emph{Saṃkṣiptābhiṣekavidhi} by Vāgīśvarakīrti. \fullcite*{sakurai1996}\mybibexclude(sakurai1996)\\

\emph{Subhāṣitaratnakośa} compiled by Vidyākara. \fullcite*{kosambigokhale1957}\mybibexclude{kosambigokhale1957}\\

\emph{Hetubinduṭīkāloka} by Durvekamiśra. \fullcite*{sanghavi1949}\mybibexclude{sanghavi1949}\\


	\section*{Secondary Sources}
	\printbibliography[notcategory=fullcited,resetnumbers,heading=none]

\end{document}
